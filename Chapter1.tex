\ifdefined\included
\else
\documentclass[english,a4paper,11pt,twoside]{StyleThese}
\include{formatAndDefs}
\sloppy
\begin{document}
\setcounter{chapter}{0} %% Numéro du chapitre précédent ;)
\dominitoc
\faketableofcontents
\fi

\chapter{From Human-Human Joint Action to Human-Robot Joint Action}
\minitoc

\section{Joint Action Theory}

What is Joint Action: sebanz06

\subsection{What do we need to perform Joint Action?}

\subsection{Commitment}

\subsection{Perception and prediction}

\subsection{Coordination}



\section{How to endow a robot with Joint Action abilities}

\subsection{Engagement and Intention recognition}

\subsection{Perspective taking and humans mental states}

\subsection{Actions realization}

\subsection{High level decisions}

\subsection{Communication for Joint Action}



\section{A three levels architecture}

\subsection{The three levels of Pacherie}

\subsection{A three levels robotics architecture}

+ related work on others robotics architecture


Cited while no other citation: \cite{goossens93}

\ifdefined\included
\else
\bibliographystyle{acm}
\bibliography{These}
\end{document}
\fi
