% Choose the language of your thesis passing 'french' or 'english' as
% \documentclass option.
% Note1: The 'page de garde' will always be written in French.
% Note2: You will have an error if you change the language of the document and
%        compile it without cleaning the auxiliary files. Compiling it again
%        should solve the problem.
\documentclass[english,a4paper,11pt,twoside]{StyleThese}
\newcommand{\included}{}
\include{formatAndDefs}
\sloppy
\begin{document}

\include{page_de_garde}

    \makeflyleaf


\dominitoc

\pagenumbering{roman}

 \cleardoublepage

% Here you can see an example of how to create text conditioned by the language
% variable. The \iftoggle command:
%
%   \iftoggle{ThesisInEnglish}{%
%   <your-text-in-english>
%   }{%
%   <your-text-in-french>
%   }
%
% will compile only one of the two blocks, depending on the variable you set at
% the beginning of this document. Language selection is managed this way in the
% formatAndDefs.tex file. You too can create sections of your thesis that is
% language dependend this way, although you probably won't need it. Another use
% of \iftoggle can be found at the end of this file.

\iftoggle{ThesisInEnglish}{%
\chapter*{Remerciements}
}{%
\chapter*{Remerciements}
}

\selectlanguage{francais}
Je tiens tout d'abord à remercier les deux personnes qui ont participé activement à l'encadrement de ma thèse,
Rachid ALAMI et Aurélie CLODIC. Ils ont toujours su me donner des conseils avisés et ont été un soutien sans faille durant ces trois années.

Je remercie Peter FORD DOMINEY et Vanessa EVERS, tout deux rapporteurs de mon manuscrit de thèse, pour m’avoir fait l’honneur d’accepter d’évaluer mon travail et de me faire bénéficier de leur expertise.
Je remercie également Raja CHATILA, Julie SHAH et Malik GHALLAB pour avoir accepté de siéger en tant qu’examinateur au sein mon jury.

Je souhaite remercier les autres doctorants, ingénieurs et stagiaires de l'équipe HRI avec qui j'ai pu travailler pour leurs idées, leur support et la bonne ambiance au sein du groupe. Merci Jules, Grégoire, Michelangelo, Harmish, Renaud, Raphaël, Camille, Yoan, Amandine, Mamoun, Erwan, Théo et Jean-François. Je remercie également Aurélie, Matthieu et Jérôme pour avoir répondu à mes nombreux appels à l'aide durant mes manipulations sur le robot.

Je remercie Amélie, Ellon, Nicolas, Laurent, Arthur, Arthur, Christophe, Rafael, Alexandre, Alejandro pour leur bonne humeur et la bonne ambiance apportée au laboratoire et en dehors ainsi que pour les parties de tarot.

Je remercie le CETAM, mon club de judo, pour leur soutien sans faille sur le tatami et en dehors sans lequel je ne serais sûrement pas arrivé au bout de cette thèse. Merci Greg, Etienne, Claire, Cyril, Olivier, Eric, Hugues, Baptise, Mimine, Jérôme, Edwin, Micka, Rachel, Yasushi et tous les autres.

Je remercie mes parents qui ont su faire de moi ce que je suis aujourd’hui.

Je tiens à remercier plus particulièrement Etienne, pour son soutien dans la vie de tous les jours durant ces trois années et ses nombreuses relectures.

Enfin, je remercie Arnaud, pour son soutien sur la fin de la thèse, son amour et bien sûr, pour l'eau du jury.

\selectlanguage{english}
\ifdefined\included
\else
\documentclass[english,a4paper,11pt,twoside]{StyleThese}
\include{formatAndDefs}
\sloppy
\begin{document}
\dominitoc
\faketableofcontents
\fi


\chapter*{Abstract}

In the future, robots will become our companions and co-workers. They will gradually appear in our environment, to help elderly or disabled people or to perform repetitive or unsafe tasks. However, we are still far from a real autonomous robot, which would be able to act in a natural, efficient and secure manner with humans. 
To endow robots with the capacity to act naturally with human, it is important to study, first, how humans act together. Consequently, this manuscript starts with a state of the art on joint action in psychology and philosophy before presenting the implementation of the principles gained from this study to human-robot joint action. We will then describe the supervision module for human-robot interaction developed during the thesis.
Part of the work presented in this manuscript concerns the management of what we call a shared plan. Here, a shared plan is a a partially ordered set of actions to be performed by humans and/or the robot for the purpose of achieving a given goal. First, we present how the robot estimates the beliefs of its humans partners  concerning the shared plan (called mental states) and how it takes these mental states into account during shared plan execution. It allows it to be able to communicate in a clever way about the potential divergent beliefs between the robot and the humans knowledge. Second, we present the abstraction of the shared plans and the postponing of some decisions. Indeed, in previous works, the robot took all decisions at planning time (who should perform which action, which object to use…) which could be perceived as unnatural by the human during execution as it imposes a solution preferentially to any other. This work allows us to endow the robot with the capacity to identify which decisions can be postponed to execution time and to take the right decision according to the human behavior in order to get a fluent and natural robot behavior. The complete system of shared plans management has been evaluated in simulation  and  with real robots in the context of a user study.
Thereafter, we present our work concerning the non-verbal communication needed for human-robot joint action. This work is here focused on how to manage the robot head, which allows to transmit information concerning what the robot's activity and what it understands of the human actions, as well as coordination signals.
Finally, we present how to mix planning and learning in order to allow the robot to be more efficient during its decision process. The idea, inspired from neuroscience studies, is to limit the use of planning (which is adapted to the human-aware context but costly) by letting the learning module made the choices when the robot is in a "known" situation. The first obtained results demonstrate the potential interest of the proposed solution.


\chapter*{Resumé}

Les robots sont les futurs compagnons et équipiers de demain. Que ce soit pour aider les personnes âgées ou handicapées dans leurs vies de tous les jours ou pour réaliser des tâches répétitives ou dangereuses, les robots apparaîtront petit à petit dans notre environnement. Cependant, nous sommes encore loin d'un vrai robot autonome, qui agirait de manière naturelle, efficace et sécurisée avec l'homme. 
Afin de doter le robot de la capacité d'agir naturellement avec l'homme, il est important d'étudier dans un premier temps comment les hommes agissent entre eux. Cette thèse commence donc par un état de l'art sur l'action conjointe en psychologie et philosophie avant d'aborder la mise en application des principes tirés de cette étude à l'action conjointe homme-robot.  Nous décrirons ensuite le module de supervision pour l'interaction homme-robot développé durant la thèse.

Une partie des travaux présentés dans cette thèse porte sur la gestion de ce que l'on appelle un plan partagé. Ici un plan partagé est une séquence d'actions partiellement ordonnées à effectuer par l'homme et/ou le robot afin d'atteindre un but donné. Dans un premier temps, nous présenterons comment le robot estime l'état des connaissances des hommes avec qui il collabore concernant le plan partagé (appelées états mentaux) et  les prend en compte pendant l'exécution du plan. Cela permet au robot de communiquer de manière pertinente sur les potentielles divergences entre ses croyances et celles des hommes. Puis, dans un second temps, nous présenterons l'abstraction de ces plan partagés et le report de certaines décisions. En effet, dans les précédents travaux, le robot prenait en avance toutes les décisions concernant le plan partagé (qui va effectuer quelle action, quels objets utiliser...) ce qui pouvait être contraignant et perçu comme non naturel par l'homme lors de l'exécution car cela pouvait lui imposer une solution par rapport à une autre. Ces travaux vise à permettre au robot d'identifier quelles décisions peuvent être reportées à l'exécution et de gérer leur résolutions suivant le comportement de l'homme afin d'obtenir un comportement du robot plus fluide et naturel. Le système complet de gestions des plan partagés à été  évalué en simulation et en situation réelle lors d'une étude utilisateur.

Par la suite, nous présenterons nos travaux portant sur la communication non-verbale nécessaire lors de de l'action conjointe homme-robot. Ces travaux sont ici focalisés sur l'usage de la tête du robot, cette dernière permettant de transmettre des informations concernant ce que fait le robot et ce qu'il comprend de ce que fait l'homme, ainsi que des signaux de coordination.
Finalement, il sera présenté comment coupler planification et apprentissage afin de permettre au robot d'être plus efficace lors de sa prise de décision. L'idée, inspirée par des études de neurosciences, est de limiter l'utilisation de la planification (adaptée au contexte de l'interaction homme-robot mais coûteuse) en laissant la main au module d'apprentissage lorsque le robot se trouve en situation "connue". Les premiers résultats obtenus démontrent sur le principe l'efficacité de la solution proposée.

\ifdefined\included
\else
\bibliographystyle{StyleThese}
\bibliography{These}
\end{document}
\fi

\tableofcontents

\printnomenclature
% Use \mtcfixnomenclature below if you have a glossary (added with
% \printnomenclature above) and you're see a shift in the mini-table of
% contents at the begining of each chapter (example: no mini-toc in chapter 1;
% mini-toc of chapter 1 appearing in chapter 2; and so on).
%
% You should not use \mtcfixnomenclature if you have no glossary (that means,
% if you don't use \printnomenclature or if your glossary is empty).
%\mtcfixnomenclature

\mainmatter

\fancyhead[RE, LO]{\bfseries\nouppercase{Introduction}}      % Chapter in the right on even pages


\ifdefined\included
\else
\documentclass[english,a4paper,11pt,twoside]{StyleThese}
\include{formatAndDefs}
\sloppy
\begin{document}
\dominitoc
\faketableofcontents
\fi


\chapter*{Introduction}
\addstarredchapter{Introduction} %Sinon cela n'apparait pas dans la table des matières
\minitoc

\section*{Context}
\addcontentsline{toc}{section}{Context}

In the 1940s, researchers invented the first machines that we can call computers. Then, they quickly came to think that this new tool which can easily manipulate numbers can also manipulate symbols and they started to work on new "thinking machines". In 1956, at the Dartmouth conference, the domain of "Artificial Intelligence" is recognized as a fully academic field. Associated to the automaton technology, the first "robots" quickly arrived in our environment.

Some of these robots are meant to work alone (e.g. rovers for space exploration) while others need to work in the vicinity and/or with humans. One possible example is robot "co-workers". These robots need to collaborate in a safe, efficient and fluent way with humans to accomplish more or less repetitive tasks. The last decades also witnessed the apparition of what is called "sociable robots" \cite{dautenhahn2007socially}. These robots can be used, for example, to help elderly or injured people in their daily life our to guide people in public spaces. 

The aim of this thesis is to make a step toward robots which act jointly with humans in a natural, efficient and fluent way. We focus more especially on the decisional issues that can appear during human-robot Joint Action. The subject of Joint Action between humans has been studied a lot in social sciences, however, many things remain to be discovered. Based on these results, the aim here is to build robots which are able to understand the humans (their beliefs and choices) and to adapt to them in order to be more pleasant and efficient companions.

\newpage
\section*{Human-Robot Joint Action challenges}
\addcontentsline{toc}{section}{Human-Robot Joint Action challenges}

Constructing robots which are able to smoothly execute Joint Action with humans brings a number of challenges. 

A first prerequisite for the robot is to be engaged in the Joint Action. If the goal is not imposed by its human partner, the robot needs to pro-actively propose its help whenever it is needed. Then, the robot needs to monitor its partner engagement in the task and demonstrate its engagement in the same task.

Once the robot has a goal to achieve, it needs to be able to find a plan to achieve this goal. This plan should be feasible in the current context of course, but it should also take the human into account, his abilities and preferences. Once a plan found, the robot should be able to share it or negotiate it with its partner. Only then, the execution of the task can begin.

During the Joint Action execution, one first challenge for the robot is to be able to understand how the humans perceive their environment and what are their knowledge concerning the task. In other words, it needs to be able to constantly estimate the mental states of its partners. These mental states should be taken into account at every steps of the execution in order to unsure a good understanding between partners.

Finally, the robot needs to be able to coordinate with the human. This coordination is needed at all level of the execution. At a lower level, the robot should exhibit an understandable and predictable behavior when performing actions. It also needs to be able to execute actions such as handover which require precise motor coordination. At a higher level, the robot needs to coordinate the Shared Plan execution. It not only needs to execute its actions at the right time but it should also give the appropriate information at the right time to its partners using either verbal or non-verbal communication.

\section*{Contributions and manuscript organization}
\addcontentsline{toc}{section}{Contributions and manuscript organization}

At the beginning of the thesis, I add to main starting points. One of these points was, as in almost all thesis I guess, the current states of the art both in robotics and in social sciences concerning human-human Joint Action. The second starting point was the current architecture for human-robot interaction developed in my research group and more especially the supervision part which I was asked to work on it. The aim with all of this being to bring innovating changes to the current system (again as in most thesis I guess). The process which I followed during this thesis and which I will explain now is graphically resumed in Fig.~\ref{fig:planIntro}.

As a first step, I studied the bibliography concerning Joint Action between humans in order to better understand what are the needed components of a successful Joint Action. I also studied the current state of the art in robotics, and more especially in human-robot interaction, in order to have an overview of what robots was already capable of. Based on all of this, I identified the needed components of a successful Human-Robot Joint Action and I studied, always based on bibliography, how to articulate them into a coherent architecture. This part constitute the Chapter~\ref{ch:biblio} of the manuscript.

Then, I took a look at the current state of the architecture for human-robot interaction developed in my research group. I especially focused on the supervision part of the architecture which I was in charge to develop. Based on the conclusion of my bibliographic study, I was able to identify several possible improvements to bring to the supervisor. The final version of the supervisor in presented in Chapter~\ref{ch:Sup} in order to help the understanding of the following of the manuscript and constitute the major technical contribution of the thesis.

\begin{figure}[!t]
	\centering
    \includegraphics[width=\textwidth]{figs/Introduction/Plan.png}
    \caption{Organization of the contributions presented in the manuscript.}
    \label{fig:planIntro}
\end{figure}

One first subject where I saw possibilities of improvement was the way the robot elaborates and executes Shared Plans. This subject is treat in the second part of the manuscript and is decomposed in three chapters:
\begin{itemize}
\item First, I noticed that there was a gap between the perspective abilities of the robot and the Shared Plan execution. Indeed, several previous works endowed the robot with the ability to estimate how its human partners perceive the world and how to use this knowledge on several domains such as dialogue. However, there was no work to link this ability to the Shared Plan execution. In Chapter~\ref{ch:MS}, I will present how we endowed the robot with the ability to estimate the humans mental states, not only about the environment, but also concerning the state of the task and more particularly of the Shared Plan. Then, I will present how the robot is able to use these mental states to better communicate about divergent beliefs during Shared Plan execution.
\item Then, coming from discussions with psychologists on the subject, we noticed that the way the robot was dealing with Shared Plans was not "natural" for humans and not enough flexible. Indeed, in the previous system, the robot was taking all decisions during Shared Plan elaboration. It was choosing for each action who should perform it and with which specific objects. Imagine a table with several identical objects on it that the robot needs to clean in collaboration with a human. The robot would have decided at the beginning for each object who should take it and in which exact order. The human would have simply removed objects and adapts to the robot decisions. The Chapter~\ref{ch:SP} aims to reduce this gap. We first identified the needed decisions during Shared Plan elaboration and execution and we endowed the robot with the ability to decide which decisions should be taken at planning time and which one are better postponed at execution time. Then, we allowed the robot to take these decision by smoothly adapting to the human choices.
\item Finally, we wanted to evaluate the wholeness of the improvements bring to the Shared Plan achievement by the robot. We did it in Chapter~\ref{ch:Eval} both quantitatively in simulation and qualitatively with a user study in the real robot. These studies allowed to show the pertinence of the proposed ameliorations and to compare two different modes developed in the context of this work (in one mode the robot negotiates some needed decisions during Shared Plan execution while in the other the robot adapts to the human choices). Moreover, for the purpose of the user study, a questionnaire has been developed to evaluate the users feelings concerning the collaboration with the robot. This questionnaire has been validated (in term of inter coherence) thanks to the study data and is generic enough to be considered as a future tool for human-robot collaboration evaluation.
\end{itemize}

Then, I saw another interesting work subject concerning the non-verbal behavior of the robot. Indeed, during Joint Action between humans, Joint Action participants exchange a lot of information through non-verbal communication. It allows to increase fluency in the task execution and to align the knowledge of all participants. Consequently, for the robot to become a better Joint Action partner, it should be able to provide such information with its non-verbal behavior. In Chapter~\ref{ch:Acting}, we studied more especially the head behavior of the robot (there is plenty other ways to give information with non-verbal behavior, but it may more need a career than a thesis to study all of them). Based on the bibliography in social sciences and on previous works in robotics, we identified needed components of a robot head behavior adapted to the Joint Action. We studied more deeply some of them with an on-line video based study. To conclude this chapter, we present how these components can be implemented into a robot head behavior architecture.

Finally, in the context of the RoboErgoSum ANR project\footnote{http://roboergosum.isir.upmc.fr/}, I have been brought to work in collaboration with ISIR at Paris where researchers work a lot in learning for robot high level decision. In Chapter~\ref{ch:Learning}, with another PhD student of ISIR Erwan Renaudo, we studied how to combine planning and learning in the context of human-robot Joint Action. The idea is to take advantage from both sides in order to come up with decision level which is able to quickly learn how to smoothly adapt to the human choices during Joint Action execution.


\section*{Work environment}
\addcontentsline{toc}{section}{Work environment}

This thesis has been realized at LAAS-CNRS in the RIS team (Robotics and InteractionS). It was included in the general objective to build a robotics architecture for an autonomous robots which interacts with humans. 

\paragraph{Robot:} in all this thesis, for practical reasons, the developed algorithms have been implemented in a PR2 robot from Clearpath Robotics (previously Willow Garage)\footnote{http://wiki.ros.org/Robots/PR2}. However, these algorithms are generic enough to be implemented in other robots. The PR2 robot is a semi-humanoid robot which is able to navigate and manipulate objects (see Fig.\ref{fig:PR2}).

\begin{figure}[!h]
	\centering
    \includegraphics[width=0.25\textwidth]{figs/Introduction/PR2.png}
    \caption{The PR2 robot.}
    \label{fig:PR2}
\end{figure}

\paragraph{Humans and objects detection:} When interacting with humans during manipulation tasks, the robot needs to be able to localize and identify humans and objects. To avoid as much as possible perceptions issues which are not the focus of this thesis, the perception of humans and objects is simplified here. The humans are identified and perceived thanks to a motion capture system. They wear a helmet to get the position and orientation of their heads and a glove to get the position and orientation of their right hands (see Fig.~\ref{fig:Environment}). Concerning the objects, they are identified and localized with tags thanks to the robot cameras in its head.

\begin{figure}[!h]
	\centering
    \includegraphics[width=0.7\textwidth]{figs/Introduction/SetUp.png}
    \caption{The PR2 robot interacting with a human to build a stack of cubes. The human is detected thanks to a motion capture system (helmet and glove) and the objects with tags.}
    \label{fig:Environment}
\end{figure}

\newpage
\section*{Publications}
\addcontentsline{toc}{section}{Publications}

The work presented in this thesis has led to several publications. They are listed here bellow (from the most recent to the oldest):
\begin{itemize}
\item \textbf{Devin, S.}, Clodic, A., Alami, R. (2017). About Decisions During Human-Robot Shared Plan Achievement: Who Should Act and How? The Ninth International Conference on Social Robotics. \textit{Submitted}.
\item \textbf{Devin, S.}, Alami, R. (2016). An implemented theory of mind to improve human-robot shared plans execution. In Human-Robot Interaction (HRI), 2016 11th ACM\/IEEE International Conference on (pp. 319-326). IEEE.
\item \textbf{Devin, S}., Milliez, G., Fiore, M., Clodic, A., Alami, R. (2016). Some essential skills and their combination in an architecture for a cognitive and interactive robot. Workshop In Human-Robot Interaction (HRI), 2016 11th ACM\/IEEE International Conference on (pp. 319-326). IEEE.
\item Khamassi, M., Girard, B., Clodic, A., \textbf{Sandra, D.}, Renaudo, E., Pacherie, E., Alami, R., Chatila, R. (2016). Integration of Action, Joint Action and Learning in Robot Cognitive Architectures. Intellectica (ARCo), 2016(65), 169-203.
\item Renaudo, E., \textbf{Devin, S.}, Girard, B., Chatila, R., Alami, R., Khamassi, M., Clodic, A. (2015). Learning to interact with humans using goal-directed and habitual behaviors. In RoMan 2015, Workshop on Learning for Human-Robot Collaboration.
\end{itemize}


\ifdefined\included
\else
\bibliographystyle{StyleThese}
\bibliography{These}
\end{document}
\fi

\setcounter{mtc}{1}

\fancyhead[RE]{\bfseries\nouppercase{\leftmark}}      % Chapter in the right on even pages
\fancyhead[LO]{\bfseries\nouppercase{\rightmark}}     % Section in the left on odd pages

\part{From Human-Human Joint Action to a supervisor for Human-Robot Interaction}

\ifdefined\included
\else
\documentclass[english,a4paper,11pt,twoside]{StyleThese}
\include{formatAndDefs}
\sloppy
\begin{document}
\setcounter{chapter}{0} %% Numéro du chapitre précédent ;)
\dominitoc
\faketableofcontents
\fi

\chapter{From Human-Human Joint Action to Human-Robot Joint Action}
\minitoc

\section{Joint Action Theory}

A first step to endow robots with the ability to perform Joint Actions with humans is to understand how humans act together. As a working definition of Joint Action, we will use the one from \cite{sebanz2006joint}:

\begin{quote}
\textit{Joint action can be regarded as any form of social interaction whereby two or more individuals coordinate their actions in space and time to bring about a change in the environment.}
\end{quote}

A given number of prerequisites are needed for these individuals to achieve the so-called Joint Action. First of all, they need to agree on the change they want to bring in the environment, the conditions under which they will stay engaged in its realisation and the way to do it. A number of works have studied this prerequisite, named \textit{commitment}, which I will develop in Sec.~\ref{subsec:commitment}. Then, as mentioned in the definition, the individuals need to coordinate their actions in space and time. This will be studied in Sec.~\ref{subsec:coordination}. Finally, in order to coordinate, each individual needs to be aware of the other, he needs to be able to perceive him and predict his actions. This part will be develop in Sec.~\ref{subsec:prediction}.

\subsection{Commitment}

\label{subsec:commitment}

The first prerequisite to achieve a Joint Action is to have a \textit{goal} to pursue and the \textit{intention} to achieve it. Let's define in a first time what is called a \textit{goal} and an \textit{intention} for a single person before going to a \textit{joint goal} and a \textit{joint intention}.

In \cite{tomasello2005understanding}, Tomasello et al. define what they call a \textit{goal} and an \textit{intention} and illustrate these definitions with an example and an associated figure (fig.~\ref{fig:intention}) where a person wants to open a box.

\begin{figure}[!h]
	\centering
    \includegraphics[width=0.8\textwidth]{figs/Chapter1/intention.png}
    \caption{Illustrative example of an intentional action by Tomasello et al. Here the human has for \textit{goal} to open the box. He chooses a means to perform it and so forms an \textit{intention}.}
    \label{fig:intention}
\end{figure}

A \textit{goal} is defined here as the representation of the desired state by the agent (in the example, the goal is an open box) and, based on Bratman's work \cite{bratman1989intention}, an \textit{intention} is defined as an action plan the agent commits itself in pursuit of a goal (in the example, the intention is to use a key to open the box). The \textit{intention} includes both a \textit{goal} and the means to achieve it. 

In a same way, Cohen and Levesque propose in \cite{cohen1991teamwork} a formal definition of what they call a \textit{persistent goal}:

\begin{quote}
\textbf{Definition: } An agent has a \textit{persistent goal} relative to \textit{q} to achieve \textit{p} iff:
\begin{enumerate}
\item she believes that \textit{p} is currently false;
\item she wants \textit{p} to be true eventually;
\item it is true (and she knows it) that (2) will continue to hold until she comes to believe either that \textit{p} is true, or that it will neither be true, or that \textit{q} is false.
\end{enumerate}
\end{quote}

However, their definition of an \textit{intention} differs a little from the previous one. They define an \textit{intention} as a commitment to act in a certain mental state:

\begin{quote}
\textbf{Definition:} An agent \textit{intends} relative to some conditions to do an action just in case she has a persistent goal (relative to that condition) of having done the action, and, moreover, having done it, believing throughout that she is doing it.
\end{quote}

The \textit{intention} still includes the \textit{goal} but here it concerns more the fact that the agent commits itself to achieve the goal than the way to achieve it.

Let's now apply these principles to a Joint Action. One of the most known definition of \textit{joint intention} is the one of Bratman \cite{bratman1993shared}:

\begin{quote}
We intend to \textit{J} if and only if:
\begin{enumerate}
\item (a) I intend that we \textit{J} and (b) you intend that we \textit{J}.
\item I intend that we \textit{J} in accordance with and because of 1\textit{a}, 1\textit{b}, and meshing subplans of 1\textit{a} and 1\textit{b}; you intend that we \textit{J} in accordance with and because of 1\textit{a}, 1\textit{b}, and meshing subplans of 1\textit{a} and 1\textit{b}.
\item 1 and 2 are common knowledge between us.
\end{enumerate}
\end{quote}

This definition is taking back and illustrated by Tomasello et al. in \cite{tomasello2005understanding} where they reuse the example of the box to open (fig~\ref{fig:intention_jointe}).

\begin{figure}[!h]
	\centering
    \includegraphics[width=0.8\textwidth]{figs/Chapter1/intention_jointe.png}
    \caption{Illustrative example of a collaborative activity by Tomasello et al. Here the humans have for \textit{shared goal} to open the box together. They choose a means to perform it which takes into account the other capabilities and so form a \textit{joint intention}.}
    \label{fig:intention_jointe}
\end{figure}

The \textit{shared goal} is defined as the representation of a desired state plus the fact that it will be done in collaboration with other person(s) (in the example, they will open the box together) and a \textit{joint intention} is defined as a collaborative plan the agents commit to in order to achieve the \textit{shared goal} and which takes into account both agents individual plans (here an agent will hold the box with the clamp while the other open it with the cutter).

In a same way, Cohen and Levesque extend their definition of \textit{persistent goal} and \textit{intention} to a collaborative activity. They first define a \textit{weak achievement goal} as:
\begin{quote}
\textbf{Definition: } An agent has a \textit{weak achievement goal} relative to \textit{q} and with respect to a team to bring about \textit{p} if either of these conditions holds:
\begin{itemize}
\item The agent has a normal achievement goal to bring about \textit{p}, that is, the agent does not yet believe that \textit{p} is true and has \textit{p} eventually being true as goal.
\item The agent believes that \textit{p} is true, will never be true, or is irrelevant (that is, \textit{q} is false), \textit{but} has as a goal that the status of \textit{p} be mutually believed by all the team members.
\end{itemize}
\end{quote}

They then use this definition to define a \textit{joint persistent goal}:
\begin{quote}
\textbf{Definition: } A team of agents have a \textit{joint persistent goal} relative to \textit{q} to achieve \textit{p} just in case
\begin{itemize}
\item they mutually believe that \textit{p} is currently false;
\item they mutually know they all want \textit{p} to eventually be true;
\item it is true (and mutual knowledge) that until they come to mutually believe that \textit{p is true}, that \textit{p} will never be true, or that \textit{q} is false, they will continue to mutually believe that they each have \textit{p} as a weak achievement goal relative to \textit{q} and with respect to the team.
\end{itemize}
\end{quote}

They finally define a \textit{joint intention} as:
\begin{quote}
\textbf{Definition:} A team of agents \textit{jointly intends}, relative to some escape condition, to do an action iff the members have a joint persistent goal relative of that condition of their having done the action and, moreover, having done it mutually believing throughout that they were doing it.
\end{quote}

As previously, the definitions of Cohen and Levesque do no take into account the way to achieve the \textit{shared goal}, however, they introduce the interesting idea that agents are also engaged to communicate about the state of the \textit{shared goal}.

Concerning the way to achieve a \textit{shared goal}, mentioned into the definition of the \textit{joint intention} of Tomasello et al., Grosz and Sidner initially introduce and formalize the notion of \textit{Shared Plan} in \cite{grosz1988plans}, which is extended in \cite{grosz1999evolution}. The key properties of their model are as follows:
\begin{quote}
\begin{enumerate}
\item it uses individual intentions to establish commitment of collaborators to their joint activity
\item it establishes an agent's commitments to its collaborating partners' abilities to carry out their
individual actions that contribute to the joint activity
\item it accounts for helpful behavior in the context of collaborative activity
\item it covers contracting actions and distinguishes contracting from collaboration
\item the need for agents to communicate is derivative, not stipulated, and follows from the general
commitment to the group activity
\item the meshing of subplans is ensured it is also derivative from more general constraints.
\end{enumerate}
\end{quote}

With their definition, each agent does not necessarily know the all \textit{Shared Plan} but only his own individual plan and the meshing subparts of the plan. The group has a \textit{Shared Plan}, but no individual member necessarily has the whole \textit{Shared Plan}.

In conclusion, the concepts concerning the commitment of agents to a collaborative activity that we will use in this thesis can be summarized as:
\begin{itemize}
\item A \textit{goal} will be represented as a desired state.
\item A\textit{shared goal} will be considered as a \textit{goal} to be achieved in collaboration with other partner(s). An agent is considered engaged in a \textit{shared goal} if he believes the goal is currently false, he wants the goal to be true and he will not give up on the goal unless he knows that the goal is achieved, not feasible or not relevant any more and he knows that his partners are aware of it.
\item A \textit{joint intention} will include a \textit{shared goal} and the way to realize it. This way will be represented as a \textit{Shared Plan} which will take into account each agent capacities and the potential conflicts between their actions. This \textit{Shared Plan} will not be necessarily completely known by all members of the group but all individuals will know their part of the plan and the meshing subparts.
\end{itemize}


\subsection{Perception and prediction}

\label{subsec:prediction}

One important thing for an agent when performing a Joint Action is to be able to perceive and predict the actions of his partner and their effects. Based on the works in \cite{sebanz2006joint}, \cite{pacherie2011phenomenology} and \cite{obhi2011moving} we identified several necessary abilities for this predictions:

\paragraph{Joint attention:} The capacity for an agent to direct his attention to where the one of his partner is directed allows to share a representation of objects and events. It brings a better understanding of the other agent knowledge and where his attention is focused and so, it helps the prediction of his possible next actions. Moreover, there should be a mutual manifestation of this joint attention, meaning that we should show that we share the other attention. 

\paragraph{Action observation:} Several studies have shown that when someone observes another person executing an action, a corresponding representation of the action is formed for the observer \cite{rizzolatti2004mirror}. This is done by what has been called the \textit{mirror-neuron} system. This behavior allows the observer to predict the outcomes of the actor's action. 

\paragraph{Co-representation:} An agent needs to have a representation of his partner, including his goal, his capacities and the social rules he is following. Indeed, having this representation will help to predict his future actions. For example, a pedestrian who sees a red traffic light will be able to predict that the car drivers will stop. In a same way, if you know that someone is aiming to go out shopping, you will be able to predict that he will look for the key of the car.


\paragraph{Agency:} Sometimes, when there is a close link between an action performed by oneself and an action performed by another, it can be hard to distinguish who caused a particular action effect. The capacity to attribute the action effects to the good actor is called the sense of \textit{Agency}. This sense of \textit{Agency} is an important thing in Joint Action in order to correctly predict the effects of each action.

\bigskip
Based on the same works as before and on \cite{sebanz2009prediction}, we can list several kinds of predictions to support Joint Action which can be done thanks to the abilities described previously :

\begin{itemize}
\item \textbf{What:} A first one is to predict what will do an agent. Two kinds of predictions can be distinguished here:
\begin{itemize}
\item \textit{action-to-goal:} this is supported by the \textit{mirror-neuron} system introduced before. Here the therm goal concerns the goal of an action, its purpose. The idea is that observing an action, it is possible to predict its goal. For example, if we observe someone extending his arm toward an object we can predict that he will pick the object.
\item \textit{goal-to-action:} here the therm goal concerns the goal of a task, as defined in the previous subsection. Knowing this goal, it can be easy to predict which action an agent will perform.
\end{itemize}
\item \textbf{When:} another prediction which is necessary is the timing of an action. Knowing when an action will occur and during how long allows to a better coordination in time.
\item \textbf{Where:} a Joint Action usually takes place in a shared space. It is therefore necessary to predict the future position of the partner and his actions in order to coordinate in space.
\end{itemize} 


\subsection{Coordination}

\label{subsec:coordination}

The predictions discussed previously allow agents to coordinate during Joint Action. Two kinds of coordination are defined in \cite{knoblich20113} that both support Joint Action.

\paragraph{Emergent coordination:}
It is a coordinated behavior which occurs unintentionally, independently of any joint plans or common knowledge and due to perception-action couplings. Four types of sources of emergent coordination can be distinguished:
\begin{itemize}
\item \textit{Entertainment:} Entrainment is a process that leads to temporal coordination of two actors’ behavior, in particular, synchronization, even in the absence of a direct mechanical coupling. It is the case, for example, for two peoples seating in rocking chairs involuntary synchronizing their rocking frequencies \cite{richardson2007rocking}.
\item \textit{Affordances:} An object affordance represents the opportunities that an object provides to an agent for a certain action repertoire \cite{gibson2014theory}. For example, the different ways to grab a mug. Two kinds of affordances can lead to an emergent coordination: \textit{common affordances} and \textit{joint affordances}. When several agents have the same action repertoire and perceive the same object they have a \textit{common affordance}. This \textit{common affordance} can lead the agents to execute the same action. When an object has affordances for two or more peoples collectively, for an action to occurs agents automatically synchronize. This is what is called \textit{joint affordances}. For example, a long two-handled saw affords cutting for two people acting together but not for either of them acting individually.
\item \textit{Perception-action matching:} As discussed before, observing an action activates corresponding representation in the observer mind. This process can lead to involuntary mimicry the observed action. Consequently, if two persons observe the same action, they can have the same reaction to mimic the action.
\item \textit{Action simulation:} The internal mechanisms activated during action observation not only allow to mimic the action but also to predict the effects of this action. If two people observe the same action and so predict the same effects, they can consequently have the same reaction. For example, two persons seeing the same object fallen will have the same reaction to try to catch it.
\end{itemize}


\paragraph{Planned coordination:} 
While emergent coordination is unintentional, planned coordination requires for agents to plan their own action in relation to Joint Action and others' actions.

One way for an agent to intentionally coordinate during Joint Action is to change his behavior compared to when he is acting alone. These changes of behavior are called \textit{coordination smoother} in \cite{vesper2010minimal} and can be of several types:
\begin{itemize}
\item Making our behavior more predictable by doing for example wider or less variable movements
\item Structuring our own task in order to reduce the need of coordination. For example sharing the space or working turn by turn.
\item Producing coordination signals like looking someone who should act or counting down.
\item Changing the way we use an object by using an affordance more appropriate to a common use.
\end{itemize}

An other way to coordinate is through communication. Indeed, Clark argues that two or more persons can not perform a Joint Action without communicate \cite{clark1996using}. Here the therm communication includes both verbal and non-verbal communication. Clark also defines what he calls the \textit{common ground}: when two agents communicate, they necessarily have common knowledge and conventions. Moreover, when communicating, it is important to not only send a message but also to assure that the message has been understood as the sender intends it to be. This process to make the sender and the receiver mutually believe that the message has been understood well enough for current purposes is called \textit{grounding}.


\section{How to endow a robot with Joint Action abilities}

In this part we will do an overview on how the theory on human-human Joint Action can be applied to human-robot Joint Action. Following to what has been discussed on commitment, we will first see in Sec.~\ref{subsec:engagement} how the robot can engage in Joint Action and understand the intention of its human partners. Then, we will see in Sec.~\ref{subsec:perspective_taking} how the robot perceives the humans and can predict their actions by taking into account their perspectives and mental states. We will also see how the robot can execute actions adapted to Joint Actions with humans in Sec.~\ref{subsec:action} and, finally, we will see how it can coordinate during Joint Action in Sec.~\ref{subsec:coordination_robot}.

The parts which are linked to the work presented in this thesis will be more developed in the corresponding chapters.

\subsection{Engagement and Intention}

\label{subsec:engagement}

As for humans, robots need to be able to engage in Joint Action. A first prerequisite is to choose a goal to perform. This goal can be imposed by a direct order of the user, however, the robot also needs to be able to pro actively propose its help whenever a human needs it. To do so, the robot needs to be able to infer high-level goals by observing and reasoning on its human partners’ activities. This process is called plan recognition or, when a bigger focus is put on Human-Robot Interaction aspects, intention recognition. Many works have been done concerning plan recognition using approaches such as classical planning \cite{ramirez2009plan}, probabilistic planning \cite{bui2003general} or logic-based techniques \cite{singla2011abductive}. Concerning intention recognition, works such as \cite{breazeal2009embodied} and \cite{baker2014modeling} take into account theory of mind aspects to deduce what the human is doing.

When direct orders have been received and humans intentions recognized, the robot needs to choose which goal to perform, also taking into account its own resources. This problem has not been addressed as a whole in the literature, however, some similar works can be seen as partial answer. For example, some deliberation systems allow to solve problems with multiple goals taking into account resources such as time \cite{georgeff1987reactive, ghallab1994representation, lemai2004interleaving} or energy level \cite{rabideau1999iterative}.

Once the robot is engaged in a Joint Action, it needs to be able to monitor other agents engagement. Indeed, it needs to understand if, for a reason, a human aborts the current goal and react accordingly. This can be done using gaze cues and gestures \cite{rich2010recognizing}, postures \cite{sanghvi2011automatic} but also context and humans mental states \cite{salam2015multi}.

Finally, once a goal chosen, the robot needs to be able to establish a Shared Plan to achieve it with its human partner(s). Several works have been done in task planning to takex into account the human \cite{cirillo2010human,Lallement2014hatp}. They allow the robot to reduce resource conflicts \cite{chakraborti2016planning}, take divergent beliefs into account \cite{guitton2012belief,talamadupula2014coordination} or promote stigmergic collaboration
for agents in co-habitation \cite{chakraborti2015planning}. 
Once the plan computed, the robot needs to be able to share/negotiate it with its partners. Several studies have been reported on how to communicate these plans. Some researchers studied how a system could acquire knowledge on plan decomposition from a user \cite{Mohseni2015} and how dialog can be used to teach new collaborative plans to the robot and to modify these plans \cite{petit2013coordinating}. In \cite{sorce2015proof}, the system is able to learn a plan from a user and transmit it to another user and in \cite{allen2002human} a computer agent is able to construct a plan in collaboration with a user. Finally, \cite{milliez2016using} presents a system where the robot shares the plan with a level of details which depends on the expertise of the user.

\subsection{Perspective taking and humans mental states}

\label{subsec:perspective_taking}

\begin{itemize}
\item Need to align robot representation of the world to humans one: x, y, theta to facts.
\item the robot needs to be able to take the perspective of its partner
\item perspective taking + basic usages (dialogue, ...)
\item but robot not only needs to take the point of view of other agents concerning the environment
\item mental states
\end{itemize}

\subsection{Actions realization}

\label{subsec:action}

\begin{itemize}
\item human-aware motion planning
\item human-aware control
\item understanding effects of actions (agency)
\item learning affordances
\end{itemize}


\subsection{Coordination}

\label{subsec:coordination_robot}

\begin{itemize}
\item plan execution
\item dialogue
\item signalling
\item joint actions (e.g. handover)
\end{itemize}

\section{A three levels architecture}

\subsection{The three levels of Pacherie}

\cite{pacherie2008phenomenology}, \cite{pacherie2011phenomenology}

\subsection{A three levels robotics architecture}

+ related work on others robotics architecture


\ifdefined\included
\else
\bibliographystyle{StyleThese}
\bibliography{These}
\end{document}
\fi

\ifdefined\included
\else
\documentclass[english,a4paper,11pt,twoside]{StyleThese}
\include{formatAndDefs}
\sloppy
\begin{document}
\setcounter{chapter}{1} %% Numéro du chapitre précédent ;)
\dominitoc
\faketableofcontents
\fi

\chapter{Supervision for Human-Robot Interaction}
\minitoc

\label{ch:Sup}

\section{Role of the supervisor in the global architecture}

One of the goal of the research group in which one I was integrated at LAAS-CNRS is to build a fully autonomous robot which interacts and performs Joint Actions with humans. To do so, an architecture for human-robot interaction has been developed and is constantly improved. This architecture is composed of several modules and a simplified scheme of it can be found in Fig.~\ref{fig:GlobalArchi}.

\begin{figure}[!h]
	\centering
    \includegraphics[width=0.7\textwidth]{figs/Chapter2/archiGlobal.png}
    \caption{The global architecture for human-robot interaction implemented at LAAS-CNRS.}
    \label{fig:GlobalArchi}
\end{figure}

\paragraph{Sensorimotor layer:}
The lower level of the architecture is composed of modules which allow to communicate and control sensors and actuators. Among others, this layer is composed of modules interpreting sensors data to detect humans and objects and a module allowing to execute given trajectories by calling the adequate actuators.

\paragraph{Situation Assessment:}
The situation assessment is done by a soft called TOASTER \cite{milliezThesis}. One of the functionalities of TOASTER is to build and maintain a consistent world state based on data coming from the sensorimotor layer. Geometrics computation are done on this world state to compute symbolic predicates concerning the environment (e.g. \textit{<object, isOn, support>}, \textit{<object, isIn, box>}) and agents abilities and behavior (e.g. \textit{object, isVisibleBy, human}, \textit{<object, isReachableBy, robot>}, \textit{human, isLookingToward, object}). TOASTER is also in charge of perspective taking: the previous predicates are contently estimated and maintained from the point of view of all humans.

\paragraph{Geometric Planner:}
In order to perform actions and movements adapted to the human proximity, our architecture is equipped with a geometric task and motion planner called GTP \cite{waldhart2016novel}. GTP allows to compute trajectories as well as objects placements and grasps in order to refine actions such as Pick or Place while taking into account the human safety and comfort.

\paragraph{Symbolic Planner:}
For the robot to be able to synthesize Shared Plans, our architecture is equipped with HATP, a human-aware HTN (Hierarchical Task Network) task planner which allows the robot to compute and refine a plan both for itself and its humans partners, taking into account a number of social rules \cite{Lallement2014hatp}.

HATP has been specially designed to integrate a number of features that are meant to promote the synthesis of plans that are acceptable by humans and easily if not trivially understandable by them. It allows to specify the humans and robot capabilities in terms of actions they can execute. Several aspects such as human preferences and comfort, estimation of human effort to achieve a task in a given context and "social rules" are used in a cost-based approach to build "sufficiently good" human-robot Shared Plans.

\paragraph{Dialogue Manager:}
In order for the robot to communicate with humans, a basic dialogue manager has been integrated to the architecture. This module allows to give humans information, ask basic questions and understand basic answers.

\paragraph{Supervisor:}
The last module of the architecture is the supervisor. It is the one in charge of controlling collaborative activities. It chooses the robot goals and monitors the Shared Plan execution. To do so, it estimates humans mental states concerning the Shared Plan and takes them into account to decide when to perform actions or to communicate (verbally and/or non-verbally). It also interprets the information coming from the Situation Assessment module in order to recognize human actions like Pick or Place with regard to the Shared Plan. This module is an extension of \cite{clodic2009shary} and \cite{fiore2016planning} and is the major technical contribution of the thesis. Its internal architecture will be detailed in the next section.

\section{The supervisor architecture}

The supervisor is composed of several modules and is fully implemented in ROS\footnote{http://www.ros.org/}. The complete scheme of its architecture can be found in Fig.~\ref{fig:archiSup}, however, as the figure is quite complex, each composing part of the supervisor will be described individually in the next subsections. 

\begin{figure}[!h]
	\centering
    \includegraphics[width=\textwidth]{figs/Chapter2/ArchiSup.png}
    \caption{Architecture of the supervisor.}
    \label{fig:archiSup}
\end{figure}

\subsection{Goal Manager}

\begin{figure}[!h]
	\centering
    \includegraphics[width=0.5\textwidth]{figs/Chapter2/GoalManager.png}
    \caption{Interaction of the Goal Manager with the rest of the supervisor.}
    \label{fig:goalManager}
\end{figure}

\subsection{Plan elaboration}

\begin{figure}[!h]
	\centering
    \includegraphics[width=0.6\textwidth]{figs/Chapter2/PlanElaboration.png}
    \caption{Interaction of the Plan Elaboration module with the rest of the supervisor.}
    \label{fig:planElaboration}
\end{figure}

\subsection{Plan Maintainer}

\begin{figure}[!h]
	\centering
    \includegraphics[width=0.8\textwidth]{figs/Chapter2/PlanMaintainer.png}
    \caption{Interaction of the Plan Maintainer with the rest of the supervisor.}
    \label{fig:planMaintainer}
\end{figure}

\subsection{Human Monitor}

\begin{figure}[!h]
	\centering
    \includegraphics[width=0.4\textwidth]{figs/Chapter2/HumanMonitor.png}
    \caption{Interaction of the Human Monitor module with the rest of the supervisor.}
    \label{fig:humanMonitor}
\end{figure}

\subsection{Mental State Manager}

\begin{figure}[!h]
	\centering
    \includegraphics[width=0.5\textwidth]{figs/Chapter2/MSManager.png}
    \caption{Interaction of the Mental State Manager with the rest of the supervisor.}
    \label{fig:MSManager}
\end{figure}

\subsection{Robot Decision}

\begin{figure}[!h]
	\centering
    \includegraphics[width=0.7\textwidth]{figs/Chapter2/RobotDecision.png}
    \caption{Interaction of the Robot Decision module with the rest of the supervisor.}
    \label{fig:robotDecition}
\end{figure}

\subsection{Action Executor}

\begin{figure}[!h]
	\centering
    \includegraphics[width=0.5\textwidth]{figs/Chapter2/ActionExecutor.png}
    \caption{Interaction of the Action Executor with the rest of the supervisor.}
    \label{fig:actionExecutor}
\end{figure}

\subsection{Non-Verbal Behavior}

\begin{figure}[!h]
	\centering
    \includegraphics[width=0.5\textwidth]{figs/Chapter2/NVBehavior.png}
    \caption{Interaction of the Non-Verbal Behavior module with the rest of the supervisor.}
    \label{fig:NVBehavior}
\end{figure}

\section{Data representation}

\begin{itemize}
\item formalization data representation
\end{itemize}

\ifdefined\included
\else
\bibliographystyle{StyleThese}
\bibliography{These}
\end{document}
\fi


\part{On the use of Shared Plans during Human-Robot Joint Action}

\ifdefined\included
\else
\documentclass[english,a4paper,11pt,twoside]{StyleThese}
\include{formatAndDefs}
\sloppy
\begin{document}
\setcounter{chapter}{2} %% Numéro du chapitre précédent ;)
\dominitoc
\faketableofcontents
\fi

\chapter{Taking Humans Mental States into account while executing Shared Plans}
\minitoc

\label{ch:MS}

\section{Motivations}

When collaborating with humans, it is primordial for the robot to not consider humans as obstacles or tools impacting the environment. As humans are social creatures, the robot must take into account their comfort and so, their point of view. Several works already allow robots to estimate humans perspective and beliefs concerning its environment. In order to improve human-robot Joint Action, the robot must be able to take these information into account when taking decision on how to act or what to communicate. Even if several works have been done on how to integrate humans perspective in dialogue or use it to help the understanding of humans behavior, there is still a gap when it came to use it during Shared Plan execution. This work aims to start filling this gap by extending the robot knowledge on humans mental states to the joint task and using it to better communicate during Shared Plan execution. It has been the subject of a publication into the HRI conference \cite{devin2016implemented}.

\section{Theory of Mind}

\subsection{Social Sciences literature}

Theory of the Mind (ToM) refers to the ability humans have to recognize and attribute mental states not only to themselves but to other people, to understand that feelings and beliefs we have may be different than others and to take others mental states into account when taking decisions. ToM has been deeply studied in psychology, notably in the developmental domain \cite{baron1985does, premack1978does}. \cite{verbrugge2008learning} defines what is called "order" of ToM:
\begin{quote}
"To have a first-order ToM is to assume that someone’s beliefs,
thoughts and desires influence one’s behavior. A first-order thought could be: ‘He does not know that his book is on the table’. In second-order ToM it is also recognized that to predict others’ behavior, the desires and beliefs that they have of one’s self and the predictions of oneself by others must be taken into account. So, for example, you can realize that what someone expects you to do will affect his behavior. For example, ‘(I know) he does not know that I know his book is on the table’ would be part of my second-order ToM. To have a third-order ToM is to assume others to have a second-order ToM, etc."
\end{quote}
There is an infinitesimal numbers of orders, however, studies shown that orders above the second one do not help in cooperative tasks \cite{de2014theory} and above the third one do not help for competitive games \cite{de2014theory}.


\begin{figure*}[!h]
    \centering
    \subfigure[Perceptual perspective taking: two individuals can have a different representation of their environment considering their locations.]{
        \centering
        \includegraphics[width=0.4\textwidth]{figs/Chapter3/perceptual.jpg}
       \label{subfig:perceptual}
   }
    %~
    \subfigure[Conceptual perspective taking: here Bob attributes to Alice a belief concerning the box. He thinks Alice thinks the box is empty.]{
        \centering
        \includegraphics[width=0.4\textwidth]{figs/Chapter3/conceptual.jpg}
       \label{subfig:conceptual}
    }
    \caption{Illustration of perceptual and conceptual perspective taking.}
\end{figure*}

ToM includes the notion of perspective taking: the capacity for a person to reason by taking the point of view of someone else. Studied in literature \cite{tversky1999speakers, flavell1992perspectives}, perspective taking is crucial during humans interaction and studies have demonstrate that individuals who lack of this ability have difficulties in their daily social interactions \cite{frick2014picturing}. Two levels of perspective taking are defined in \cite{flavell1977development}: perceptual and conceptual perspective taking. Perceptual perspective taking design the capacity of a person to understand that others have a different perception of the world (fig~\ref{subfig:perceptual}). Conceptual perspective taking designs the capacity of a person to attribute beliefs and feelings to others (fig~\ref{subfig:conceptual}).

\begin{figure}[!h]
	\centering
    \includegraphics[width=0.7\textwidth]{figs/Chapter3/sally.jpg}
    \caption{The Sally and Anne test: it allows to check the capacity of someone to attribute a false-belief to another person. Illustration from the work of Axel Scheffler.}
    \label{fig:sally}
\end{figure}

To check if an individual has ToM capacities, several tests have been developed in psychology. One of the most famous is the Sally and Anne test (fin~\ref{fig:sally}). This test allows to check the capacity of someone to attribute a false-belief to another person and have been reused in robotics to validate robots perspective taking abilities.

\subsection{Robotics background}

One of the pioneer work in robotics Theory of Mind is \cite{scassellati2002theory}. Scassellati presents two models from social sciences (Leslie \cite{leslie1984spatiotemporal} et Baron-Cohen \cite{baron1997mindblindness}) and proposes a model on how to implement ToM in robotics. However, the implementation of this model did not go further than perception level.

Then, several works have been done in order to endow robots with perspective taking abilities. Using ACT-R architecture \cite{anderson2004integrated}, the team of Hiatt and Trafton models mechanisms used during the Sally and Anne test and constructs a model that learns to deal with false belief in order to pass this test \cite{hiatt2010cognitive}. They extend this work to second-order in \cite{hiatt2015understanding} and to spatial reasoning in \cite{hiatt2004cognitive}. The Sally and Anne test has also been passed in \cite{milliez2014framework} where the robot constructs a semantic representation of the world from its partners point of view. In \cite{berlin2006perspective}, authors present a way to record different beliefs of other agents and so to have a memory of perspective taking. Finally, \cite{johnson2005perspective} presents a system which computes perspective taking based on forward and inverse visual models.

Perspective taking abilities have been used in robotics for several purposes. It has been used in \cite{hiatt2011accommodating} to deal with uncertainty in humans behavior and in \cite{ros2010solving} to solve ambiguous references to an object. One important application of perspective taking is action recognition. \cite{johnson2005perceptual} takes the visual point of view of humans to improve action recognition, Dynamic Bayesian Networks (DBN) are used in \cite{baker2014modeling} or inverse reinforcement learning  in \cite{nagai2015probabilistic}. The human perspective is also used in \cite{breazeal2006using} to learn a task from a situation that can be ambiguous from the robot point of view and in \cite{gray2014manipulating} to choose actions with the adequate effects in order to manipulate humans mental models. Finally, \cite{gorur2017toward} uses perspective taking to infer humans intention and adapt robot decision.

Concerning Shared Plans, perspective taking can be used to help their elaboration in order to add communication actions to solve divergent beliefs \cite{guitton2012belief}. Then, the human perspective is used to share the plan with a level of details depending of human knowledge \cite{milliez2016using}. However, there is no works for now concerning the management of Shared Plans execution taking into account the human point of view. 

\section{Assumptions}


The work presented in this chapter concerns the estimation of humans knowledge on the task and its use to help the Shared Plan execution. To do so, we make several assumptions:

\paragraph{Commitment:} we do not focus in this works on issues related to commitment. Consequently, we consider here that the joint goal has already been established. We also consider that none of the humans will abort the goal unless he knows that the goal is not achievable any more.

\paragraph{Shared Plan:} the focus of this work concerns rather the Shared Plan execution than the Shared Plan elaboration. In the examples presented in Sec.~\ref{sec:resultsTOM}, the Shared Plan is computed by the robot, however, the processes presented in this chapter hold regardless of the way the robot gets the Shared Plan (e.g. it can be imposed by a human or negotiated through dialogue). This chapter will treat only the issues related to ToM usage in Shared Plan execution, the rest of Shared Plan management will be more developed in Chapter \ref{ch:SP}.

\paragraph{ToM order:} this work implements a first-order ToM for the robot (i.e. the robot has knowledge about the human knowledge on the task), the higher orders are not managed for now.

\paragraph{Humans perception:} we make the assumption here that a human will see and understand an action of another agent (mainly robot actions) when he is present and looking at the agent. We
also assume that when he is present, the human is able to hear and understand the information verbalized by the robot.

\paragraph{Robot capacities:} we consider that the robot is able to perform simple high level actions like Pick, Place or Drop. We also assume that the robot is able to ask to a human to perform an action and to inform him about the state of the environment, the goal or an action. The robot is able to detect and localize objects and agents
and to recognize simple high level actions performed by a human like Pick, Place or Drop. Let us also note that the ways the robot achieves actions (e.g. human-aware motion planning and execution) and recognizes humans’ actions are outside of the scope of this chapter.

\paragraph{Communication:} this work consists mainly in finding which information to give to the human and at which time. We do not focus here on the how we give these information (here we use the basic dialogue module described in Chapter~\ref{ch:Sup} but more complex communication mechanisms can be envisioned). 

\section{Estimating Humans Mental States}

As stated previously, the goal of this work is to fill the gap between existing perspective taking abilities of the robot and Shared Plan execution. A first step to do so is to extend the knowledge of the robot concerning humans mental states to information concerning the Shared Plan. As saw in Chapter~\ref{ch:Sup}, the mental state of a human H will be described as:
$$MS(H) = <g_H, g_R(H), SP(H), WS(H)>$$
where $g_H$ is the goal the robot estimates the human is engaged in, $g_R(H)$ is the goal the robot estimates the human thinks the robot is performing, and $SP(H)$ and $WS(H)$ are the estimation of the Shared Plan and the World State from the human point of view. 

The process to estimate the humans mental states will be noted in the following of the thesis as the operator:
$$MS(H) \leftarrow ESTIMATE\_MS(MS(H), TS)$$

with $TS$ the state of the task from the robot point of view as stated in Chapter~\ref{ch:Sup}.
We will see now how we estimates each of the mental states components.

\subsection{Goal management ($g_H$ and $g_R(H)$ computation)}

As stated previously, the focus of this work is not put on the goal management. Consequently, the computation of humans mental states concerning goals remains basic. However, a more complex one can be envisioned, for example using intention recognition, with the same representation. As a reminder, a goal is defined as:
$$g = <Name_g, Actors_g, Params_g, Obj_g>$$

As we consider humans automatically engaged in the goal, as soon as the robot starts executing a goal, all actors of the goal are considered to have the same goal:
$$ \forall H \in Actors_{g_R}, \ g_H = g_R$$
We also make the basic assumption that all humans who see the robot are aware of its goal:
$$ \forall H  \ | <Robot, isVisibleBy, H> \in WS, \ g_R(H) = g_R$$
For a goal to be considered achieved by an agent (it holds for human mental states as well as the robot mental state), this agent needs to have all the objectives of the goal in its knowledge (it means that accordingly to its knowledge, the desired world state has been reached):
$$ \forall H, \ \forall g  \ | \ Obj_g \in WS(H), \ label_g = DONE$$
The robot will consider a goal failed if it does not find a plan any more to achieve it. Concerning the humans, they can be informed through dialogue by the robot of the failure (or success) of a goal.

\subsection{Shared Plan management ($SP(H)$ computation)}

As a reminder, the representation of the Shared Plan $SP$ from a human H point of view is represented as:
$$SP(H) = <id_p(H), A_p(H), L_p(H)>$$
where $id_p(H)$ is used to identify the plan, $A_p(H)$ are the actions composing the plan and $L_p(H)$ the links representing the order the actions should be executed (causal links).

As we consider in this thesis Shared Plans with action allocation evolving during the execution, we made the choice not to share the plan and communicate about actions only when it is not implicit (more details in Chapter~\ref{ch:SP}). Hence, we consider that the Shared Plan of the human is always the same as the robot one, only the state of the actions composing the plan will change:
$$SP(H) = <id_p, A_p(H), L_p>$$

A link $l \in L_p$ can be describe as:
 $$l = \langle prev_l, \ next_l \rangle$$
where $prev_l$ is the id of the action which needs to be achieved before the action with the id $next_l$ is performed. 

The actions composing the plan $A_p(H)$ can be decomposed as:
$$A_p(H) = <A_{prev}(H), A_{cur}(H), A_{next}(H), A_{later}(H)>$$
where $A_{prev(H)}$ are the actions of the plan the human thinks already executed, $A_{cur}(H)$ the actions the human thinks currently executed, $A_{next}(H)$ the actions the human thinks which can be performed and $A_{later}(H)$ the actions the human thinks to be executed in the future. Each action $a$ in $A_{prev}(H)$ is associated with a label noted $label_a$ which can be equal either to DONE, FAILED or ABORTED.

By default, when a Shared Plan is computed by the robot, all actions are put in $A_{later}(H)$. When the robot performs an action or detects an action execution from a human, it considers the human is aware of the action if he can see the actors of the action:

\begin{center}
$a \in A_{cur} \ \& \ (<Ag_a, isVisibleBy, Human> \in WS \ \| \ Human \in Ag_a)$ 

$\Rightarrow a \in A_{cur}(H)$
\end{center}

In a same way, at the end of the execution, the action goes in $A_{prev}(H)$ with the label corresponding to the success or the failure of the action if the human performs or saw the actors of the actions at the end of the action.

We also consider that a human can infer that an action has been done if it knows that the action was in progress or on its way to be done and it can see the effects of the action:

\begin{center}
$(a \in A_{cur}(H) \ \| \ a \in A_{next}(H)) \ \& \ Effects_{a} \in WS(H)$ 

$\Rightarrow a \in A_{prev}(H) \ \& \ label_a(H) = DONE$
\end{center}

In a same way, we consider that if a human knows that an action was in progress and can see the actors of the action while there are not performing the action any more, he considers the action DONE:

\begin{center}
$a \in A_{next}(H) \ \& \ a \in A_{prev} \ \& \ <Ag_a, isVisibleBy, Human> \in WS$

$\Rightarrow a \in A_{prev}(H) \ \& \ label_a(H) = DONE$
\end{center}

Finally, the actions are set in $A_{next}(H)$ considering causal links and preconditions:

\begin{center}
$a \in A_{next}(H) \Leftrightarrow Precs_{a} \in WS(H) \ \& \ (\forall l \in L_p$ 

$| \ next_l = id_a, \exists \ ap \in A_{prev}(H) \ | \ (id_{ap} = prev_l \ \& \ label_{ap}(H)  = DONE))$
\end{center}

The robot can also inform a human about the state of an action, in which case the given information will be add to the human mental state.

\subsection{World State management ($WS(H)$ computation)}
\label{subsec:worldstate}

We saw that the perspective taking abilities of the robot allow it to estimate the human perception of his environment \cite{milliez2014framework}. However, this previous work only concerns information about the environment which are perceivable. Indeed, for this work we will consider two kinds of predicates to describe the state of the environment:
\begin{itemize}
\item \textbf{Observable predicates:} they concern what the agent can observe about the world state. These predicates mainly represent the affordances of all agents (e.g. isVisibleBy, isReachableBy) and the relations between objects (e.g. isOn, isIn) visible to them. They are computed continuously by the Situation Assessment module (TOASTER) from the robot and humans point of views based on geometric computations and perspective taking algorithms.
\item \textbf{Non-observable predicates:} they concern information that the agent can not observe (e.g. the fact that an opaque bottle is empty). These predicates are not managed by TOASTER which reason on what is visible by the agents. We consider two ways for an agent to be aware of a non-observable predicate. First, it can perform or see an action which has this predicate in its side effects:
$$\forall a \ \in \ A_{prev} \ | \ label_{a} \ = \ DONE \Rightarrow Effects_{a} \rightarrow WS$$
(likewise with $A_{prev}(H)$ and $WS(H)$). A human can also be aware of a non-observable predicate if he is informed of it by the robot.
\end{itemize}


\section{Mental States for Shared Plans execution}

We saw in the previous section how we estimate humans mental states concerning the shared task. We will see now how we use them to communicate during the Shared Plan execution. Indeed, when two humans share a plan, they usually do not communicate all along the plan execution. Only the \textit{meshing subplans} of the plan need to be shared \cite{bratman1993shared}. Consequently, the robot should inform humans about elements of the shared plan only when it considers that the divergent belief might have an impact on the joint activity in order to not be intrusive by giving them information which they do not need or which they can observe or infer by themselves. The process of monitoring the divergent beliefs and solving them if needed which will be described in this section will be noted in the following of the thesis as:
$$SOLVE\_DB(MS(H), TS)$$

\subsection{Weak achievement goal}

If we follow the definition of \textit{weak achievement goal} in \cite{cohen1991teamwork}, if the robot knows that the current goal has been achieved or is not possible anymore, it has to inform its partners. Accordingly, we consider that, when, in the robot knowledge, the label of a goal is DONE (resp. ABORTED) and the robot estimates that a human does not consider it DONE (resp. ABORTED), the robot informs him about the achievement (resp. abandoning) of the goal (if the agent is not here or is busy with something else, the robot will do it as soon as the agent is available).

\begin{algorithm}
\caption{Weak achievement goal}
\label{alg:informGoal}
\begin{algorithmic}
\IF {$\exists \ g \ | \ (label_g = DONE \ \& \ label_g(H) \neq DONE) \ \| \ (label_g = ABORTED \ \& \ label_g(H) \neq ABORTED)$ \hfill \textit{$\vartriangleright$ There is a divergent belief to solve}
\STATE}
\STATE Inform($g$)\hfill \textit{$\vartriangleright$ The robot informs the human about the state of the goal}
\ENDIF
\end{algorithmic}
\end{algorithm} 

\subsection{Before humans action}

A divergent belief of a human partner can be an issue when it is related to an action that he has to perform. To avoid that a human misses information to execute his part of the Shared Plan, each time the robot estimates that a human has to perform an action (action in $A^H_{next}$) it checks if the human is aware that he has to and can perform the action (the action should also be in $A^H_{next}(H)$).
In order not to give too much information at the same time, and as not yet allocated actions can also be performed by the robot, the robot checks for actions in $A^X_{next}(H)$ only if the human does not have any other actions to perform. If there is a divergent belief, there are two possible reasons:
\begin{itemize}
\item The human misses information about previous achieved actions to know that his action has to be performed now according to the plan. The robot checks the label of all actions linked to the first one with the plan links and informs about the achievement of all actions with a label different of DONE in the estimation of the human knowledge.
\item The human misses information about the world state to know that his action is possible. In such case, the robot looks into the preconditions of the actions and informs the human about all those the human is not aware of.
\end{itemize}
The given algorithm to solve this kind of divergent belief is summarized in Alg.~\ref{alg:checkHumanAction}.


\begin{algorithm}
\caption{Checking humans actions}
\label{alg:checkHumanAction}
\begin{algorithmic}
\IF {$(\exists \ action \in A^H_{next} \ | \ action \notin A^H_{next}(H)) \ \| \ (A^H_{next} = \emptyset \ \& \ \exists \ action \in A^X_{next} \ | \ action \notin A^X_{next}(H))$}
\STATE \hfill \textit{$\vartriangleright$ There is a divergent belief to solve}
\IF {$\exists \ a \in A_{p} \ | \ (\exists \ l \in L_p \ | \ next_l = action \ \& \ prev_l = a) \ \& \ (a \notin A_{prev}(H) \ \| \ label_a(H) \neq DONE))$}
\STATE Inform($a$)\hfill \textit{$\vartriangleright$ The robot says to the human that the action is done}
\ENDIF
\IF {$\exists p \in Precs_{action} \ | \ p \notin WS(H))$}
\STATE Inform($p$)\hfill \textit{$\vartriangleright$ The robot informs about the missing precondition}
\ENDIF
\ENDIF
\end{algorithmic}
\end{algorithm} 

\subsection{Preventing mistakes}

A divergent belief of a human partner can also be an issue if it leads him to perform an action that should not be perform now according to the plan. To prevent this, for each action that the robot estimates the human thinks he has to execute (action in $A^H_{next}(H)$), the robot checks if the action really needs to be performed (the action should also be in $A^H_{next}$). In the same way, the robot also checks the actions not yet allocated as the human can perform them (action in $A^X_{next}(H)$ which is not in $A^X_{next}$). If there is a divergent belief, the robot corrects the human divergent belief by two different ways:
\begin{itemize}
\item The human can think that a previous action has been achieved successfully while it is not the case leading him to think he has to perform another action. The robot looks in all actions linked to the first one by the plan links and informs about their state if it is different in the estimation of the human knowledge and in the robot knowledge.
\item The human can have a divergent belief concerning the world state that leads him to think that his action is possible while it is not the case. The robot looks into the preconditions of the action and informs about divergent beliefs.
\end{itemize}
The given algorithm to solve this kind of divergent belief is summarized in Alg.~\ref{alg:checkMistakes}.


\begin{algorithm}
\caption{Preventing mistakes}
\label{alg:checkMistakes}
\begin{algorithmic}
\IF {$\exists \ action \in {A^H_{next}(H) \ \cup \ A^X_{next}(H)} \ | \ action \notin \{A^H_{next} \ \cup \ A^X_{next}\}$}
\STATE \hfill \textit{$\vartriangleright$ There is a divergent belief to solve}
\IF {$\exists \ a \in A_{p} \ | \ (\exists \ l \in L_p \ | \ next_l = action \ \& \ prev_l = a) \ \& \ (label_a(H) = DONE) \ \& \ (a \notin A_{prev} \ \| \ label_a \neq DONE))$}
\STATE Inform($a$)\hfill \textit{$\vartriangleright$ The robot informs about the state of the action}
\ENDIF
\IF {$\exists p \in Precs_{action} \ | \ p \in WS(H) \ \& \ p \notin WS)$}
\STATE Inform($p$)\hfill \textit{$\vartriangleright$ The robot informs about the wrong precondition}
\ENDIF
\ENDIF
\end{algorithmic}
\end{algorithm} 


\subsection{Signal robot actions}

When the robot is about to perform an action, it checks if it estimates that the humans are aware that it will act (the action should also in $A^R_{next}(H)$). If it is not the case, the robot signals its action before performing it (Alg.~\ref{alg:signalActions}).

\begin{algorithm}
\caption{Signal robot actions}
\label{alg:signalActions}
\begin{algorithmic}
\IF {$\exists \ action \in A^R_{next} \ | \ action \notin A^R_{next}(H)$ \hfill \textit{$\vartriangleright$ There is a divergent belief to solve}
\STATE}
\STATE Signal($action$)\hfill \textit{$\vartriangleright$ The robot signals its action}
\ENDIF
\end{algorithmic}
\end{algorithm} 


\subsection{Inaction and uncertainty}

Even if the robot estimates that the human is aware that he has to act (there is action(s) in $A^H_{next}(H)$), it is possible that the human still does not perform his action(s). If the human is already busy by something else (there is an action in $A^H_{cur}$) or if it is not currently engaged in the task, the robot waits for the human to be available. If the human is not considered busy by the robot, the robot first considers that its estimation of the human mental state can be wrong, and that, in reality, the human is not aware that he should act. Consequently, the robot asks the human specifically to perform the action. If the human still does not act while the action has been asked, the robot considers the action failed, aborts the current plan and tries to find an alternative plan excluding that action.


\section{Results}

\label{sec:resultsTOM}

\subsection{Tasks}

In order to evaluate the benefits of our method during human-robot interaction, we will use two different tasks. We will first show an illustrative example of one possible scenario of the first task, and then, we will run simulations on these two tasks in order to get objective measurement of the performance of the system in simulation. The system will be evaluated in a real situation latter in Chapter~\ref{ch:Eval}.

\paragraph{"Clean the table" scenario}

In this example, a PR2 robot and a human have to clean a table together. To do so, they need to remove all items from this table, sweep it, and re-place all previous items. The initial world state is the one in Fig.~\ref{fig:initClean}. We consider that the grey book is reachable only by the robot, the blue book only by the human and the white book by both agents. The human and the robot have the ability to pick objects and place them into another support. Only the robot has the capacity to sweep the table. The initial plan produced to achieve the goal is shown in Fig.~\ref{fig:initPlanClean}.

\begin{figure}[!h]
	\centering
    \includegraphics[width=0.6\textwidth]{figs/Chapter3/cleanWithNames.png}
    \caption{Initial situation of the "Clean the table" scenario.}
    \label{fig:initClean}
\end{figure}

\paragraph{"Inventory" scenario}

In this example, a human and a PR2 robot have to make an inventory together. At the beginning of the task, both agents have coloured objects near them as well as a coloured box (initial world state in Fig.~\ref{fig:initInventory}). These coloured objects need to be scanned and then, stored in the box of the same colour. To do so, both agents can pick objects and place them in the table in a way reachable by the other agent. They also both have the ability to drop objects in the box near them. Finally, only the robot can scan an object, it consists of orienting its head and turning on a red light in the direction of a reachable object. 

\begin{figure}[!h]
	\centering
    \includegraphics[width=0.6\textwidth]{figs/Chapter3/initInventory.png}
    \caption{Initial situation of the "Inventory" scenario.}
    \label{fig:initInventory}
\end{figure}



\subsection{Illustrating scenario}

We will first illustrate the benefits of the work presented in this chapter with an example. This example is based on the "Clean the table" task presented previously. At the beginning of the interaction the robot computes the plan Fig.~\ref{fig:initPlanClean}. 

\begin{figure}[!h]
	\centering
    \includegraphics[width=0.7\textwidth]{figs/Chapter3/InitCleanPlan.png}
    \caption{Initial plan of the "Clean the table" scenario. }
    \label{fig:initPlanClean}
\end{figure}


The robot starts to pick and place the grey book on the light-coloured shelf. The human picks and places the blue book on the dark-coloured shelf then leaves (Fig.~\ref{subfig:humanLeave}).

\begin{table}[!h]
\begin{center}
\begin{tabular}{|c|c|c|c||c|c|c|c|}
\hline
\multicolumn{4}{|c||}{Robot} & \multicolumn{4}{c|}{Human}\\
\hline
$A_{prev}$ & $A_{cur}$ & $A_{next}$ & $A_{ready}$ & $A_{prev}$ & $A_{cur}$ & $A_{next}$ & $A_{ready}$\\
\hline
\hline
1 & 0 & 2 & 3, 4, 5, 6 & 1 & 0 & 2 & 3, 4, 5, 6\\
\hline
\end{tabular}
\end{center}
\caption{Knowledge of the robot and estimation of the human knowledge after the human left. The numbers represent the actions id as stated in the plan Fig.~\ref{fig:initPlanClean}.}
\label{table:results}
\end{table}

The robot ends its action. At this point, the only possible action of the plan has to be done by the human. The robot waits a few time for the human and then, as the human does not come back (and so does not execute its action), aborts the current plan and computes a new one where it removes the last book (Fig.~\ref{fig:newplan}).
The robot picks and places the white book on the light-coloured shelf and sweeps the table (Fig.~\ref{subfig:robotSweep}).

\begin{figure}[!h]
	\centering
    \includegraphics[width=0.7\textwidth]{figs/Chapter3/SecondCleanPlan.png}
    \caption{Second plan of the "Clean the table" scenario.}
    \label{fig:newplan}
\end{figure}

\begin{table}[!h]
\begin{center}
\begin{tabular}{|c|c|c|c||c|c|c|c|}
\hline
\multicolumn{4}{|c||}{Robot} & \multicolumn{4}{c|}{Human}\\
\hline
$A_{prev}$ & $A_{cur}$ & $A_{next}$ & $A_{ready}$ & $A_{prev}$ & $A_{cur}$ & $A_{next}$ & $A_{ready}$\\
\hline
\hline
0, 1, 7, 8  &  & 9, 11 & 10 & 1 & 0 & 7 & 8, 9, 10, 11\\
\hline
\end{tabular}
\end{center}
\caption{Knowledge of the robot and estimation of the human knowledge after the robot swept the table. The numbers represent the actions id as stated in the plan Fig.~\ref{fig:newplan}.}
\label{table:results}
\end{table}

The human comes back at this time (Fig.~\ref{subfig:humanComesBack}). As he can see that the grey book is on the shelf near the robot, the robot infers that he infers that the robot has achieved the action it was performing when the human left. Moreover, as the human can see that the white book is on the shelf near the robot, the robot infers that he infers that the robot moved the book. However, the human can not observe that the table has been swept by the robot (we consider here that the effects of the \textit{sweep} action are not observable). Consequently, the human does not know that he can puts back the book he removed.

\begin{table}[!h]
\begin{center}
\begin{tabular}{|c|c|c|c||c|c|c|c|}
\hline
\multicolumn{4}{|c||}{Robot} & \multicolumn{4}{c|}{Human}\\
\hline
$A_{prev}$ & $A_{cur}$ & $A_{next}$ & $A_{ready}$ & $A_{prev}$ & $A_{cur}$ & $A_{next}$ & $A_{ready}$\\
\hline
\hline
0, 1, 7, 8  &  & 9, 11 & 10 & 0, 1, 7 &  & 8 & 9, 10, 11\\
\hline
\end{tabular}
\end{center}
\caption{Knowledge of the robot and estimation of the human knowledge when the human comes back. The numbers represent the actions id as stated in the plan Fig.~\ref{fig:newplan}.}
\label{table:results}
\end{table}


As the robot estimates that the human does not know that he has to put back the book he removed, it uses its knowledge on the plan to deduce that it is because the human does not know that the table has been swept. So, the robot informs the human about this (by verbalization). 

\begin{table}[!h]
\begin{center}
\begin{tabular}{|c|c|c|c||c|c|c|c|}
\hline
\multicolumn{4}{|c||}{Robot} & \multicolumn{4}{c|}{Human}\\
\hline
$A_{prev}$ & $A_{cur}$ & $A_{next}$ & $A_{ready}$ & $A_{prev}$ & $A_{cur}$ & $A_{next}$ & $A_{ready}$\\
\hline
\hline
0, 1, 7, 8  &  & 9, 11 & 10 & 0, 1, 7, 8 &  & 9, 11 & 10\\
\hline
\end{tabular}
\end{center}
\caption{Knowledge of the robot and estimation of the human knowledge after the robot informed the human. The numbers represent the actions id as stated in the plan Fig.~\ref{fig:newplan}.}
\label{table:results}
\end{table}

The human has know all the information he needs to finish the task. The robot and him both perform their last actions and so achieve the task (Fig.~\ref{subfig:endClean}).


\begin{figure*}[!h]
    \centering
    \subfigure[The human leaves before removing his first book from the table.]{
        \centering
        \includegraphics[width=0.4\textwidth]{figs/Chapter3/HumanLeave.png}
       \label{subfig:humanLeave}
   }
    %~
    \subfigure[The robot removes the last book and sweep the table.]{
        \centering
        \includegraphics[width=0.44\textwidth]{figs/Chapter3/robotSweep.png}
       \label{subfig:robotSweep}
    }
    %~
    \subfigure[The human comes back.]{
        \centering
        \includegraphics[width=0.4\textwidth]{figs/Chapter3/humanComesBack.png}
       \label{subfig:humanComesBack}
   }
    %~
    \subfigure[The human and the robot perform their last actions and achieve the task.]{
        \centering
        \includegraphics[width=0.45\textwidth]{figs/Chapter3/endClean.png}
       \label{subfig:endClean}
    }
    \caption{Illustrative "Clean the table" scenario.}
\end{figure*}

\subsection{Experiment and results}

We will now evaluate the benefits of the presented work in simulation in the two tasks described previously. Results in real situations as well as more simulation results with the whole system developed in the thesis can be found in Chapter~\ref{ch:Eval}). The results here only concern the use of mental states during Shared Plan execution.

When the interaction starts, we consider that the joint goal is already established and that a first Shared Plan has been computed by the robot. The robot executes the plan and the simulated human executes the actions planned for him. We randomly sample a time when the human leaves the scene and another time when the human comes back. While absent, the human does not execute actions and cannot see anything nor communicate.

One objective of our contribution is to reduce unnecessary communication from the robot during the execution of a Shared Plan aiming at a more friendly and less intrusive behaviour of the robot. Consequently, in order to evaluate our system, we have chosen to measure the amount of information shared by the robot during a Shared Plan execution. During the interaction, we logged the number of facts (information chunks) given by the robot to the human. An information concerns either a change in the environment or the state of a previous action. 

We compared our system (called \textit{ToM system}) to:
\begin{itemize}
\item a system which informs about each action missed by the human (called \textit{Missed system}).
\item a system informs about each action performed by the robot even if the human sees it (called \textit{Performed system}).
\end{itemize}

The obtained results in 100 runs are given in Table~\ref{table:results}.

\begin{table}[ht]
\begin{center}
\begin{tabular}{|r||c|c||c|c|}
\hline
 Scenario & \multicolumn{2}{c||}{Clean the table} & \multicolumn{2}{c|}{Inventory}\\
\cline{2-5} 
System & Average & Std Dev & Average & Std Dev\\
\hline
\hline
ToM & 0.94 & 0.24 & 0.41 & 0.48\\
\hline
Missed & 2.14 & 0.87 & 2.61 & 1.36\\
\hline
Performed & 3.72 & 0.96 & 10.0 & 0.0\\
\hline
\end{tabular}
\end{center}
\caption{Number of information given by the robot during the two presented scenarios for the three systems (\textit{TOM}, \textit{Missed} and \textit{Performed}).}
\label{table:results}
\end{table}

We can see that our system allows to reduce significantly the amount of information given by the robot. In the "Clean the table" scenario, depending on when the human leaves, the robot might change the initial plan and take care of the book reachable by both agents instead of the human. This explains why the average number for the \textit{Performed system} is higher than the number of actions initially planned for the robot: the robot performs more actions in the new plan. In this scenario, our system allows not to communicate about missed \textit{pickandplace} actions as the human can infer them by looking at the objects placements. However, the robot will inform the human if he missed the fact that the robot has swept the table as it is not observable and it is a necessary information for the human to know before he can put back objects on the table.

In the inventory scenario, as all objects and boxes are reachable only by one agent, the robot does not change the plan when the human leaves. This explains the fact that the standard deviation is null for the \textit{Performed system}: the number of actions performed by the robot never changes and there is no change in the plan. In this scenario, the \textit{pickanddrop} and \textit{scan} actions have non-observable effects (the human can not see an object in a box). However, we can see that our system still verbalizes less information than the \textit{Missed system}: the robot communicates only the information which the human really needs (as the fact that an object the human should drop in a box has been scanned) and does not give information which are not linked to the human part of the plan (as the fact that the robot scanned an object it have to drop in its box or that the robot dropped an object).


\section{Conclusion}

In this chapter we shown how we extended the robot estimation of humans mental states (which initially concerned only the environment) to the state of the task and more specifically of the Shared Plan. Then, we shown how we use these mental states to better communicate during Shared Plan execution.

The benefits of this work have been demonstrated with an illustrative example and simulation results. These results show that the proposed system allows to reduce communication by removing useless information given by the robot.

The next chapter will give more details about Shared Plan elaboration and management, more particularly concerning actions allocation. Then, we will show in Chapter~\ref{ch:Eval}) more complete simulation results with the whole system as well as results with the system running in a real situation.

\ifdefined\included
\else
\bibliographystyle{StyleThese}
\bibliography{These}
\end{document}
\fi

\ifdefined\included
\else
\documentclass[english,a4paper,11pt,twoside]{StyleThese}
\include{formatAndDefs}
\sloppy
\begin{document}
\setcounter{chapter}{3} %% Numéro du chapitre précédent ;)
\dominitoc
\faketableofcontents
\fi

\chapter{When to take decisions during Shared Plans elaboration and execution}
\minitoc

\section{Background}

\section{Main principle}

\section{Shared Plans elaboration}

\section{Shared Plans execution}

\section{Results}

In simulation

\ifdefined\included
\else
\bibliographystyle{acm}
\bibliography{These}
\end{document}
\fi

\ifdefined\included
\else
\documentclass[english,a4paper,11pt,twoside]{StyleThese}
\include{formatAndDefs}
\sloppy
\begin{document}
\setcounter{chapter}{4} %% Numéro du chapitre précédent ;)
\dominitoc
\faketableofcontents
\fi

\chapter{Acting during Shared Plans Execution}
\minitoc

\section{Actions instantiation: which object to use?}

\section{Execution of collaborative actions}

TODO: on Pepper

\section{Head signaling}

TODO + biblio


\ifdefined\included
\else
\bibliographystyle{StyleThese}
\bibliography{These}
\end{document}
\fi


\part{Other contributions to Human-Robot Joint Action}

\ifdefined\included
\else
\documentclass[english,a4paper,11pt,twoside]{StyleThese}
\include{formatAndDefs}
\sloppy
\begin{document}
\setcounter{chapter}{5} %% Numéro du chapitre précédent ;)
\dominitoc
\faketableofcontents
\fi

\chapter{Evaluation of the global system}
\minitoc

\label{ch:Eval}

\section{Scenario}

\section{Evaluation in simulation}

\section{User study}


\ifdefined\included
\else
\bibliographystyle{StyleThese}
\bibliography{These}
\end{document}
\fi

\ifdefined\included
\else
\documentclass[english,a4paper,11pt,twoside]{StyleThese}
\include{formatAndDefs}
\sloppy
\begin{document}
\setcounter{chapter}{6} %% Numéro du chapitre précédent ;)
\dominitoc
\faketableofcontents
\fi

\chapter{Combining learning and planning}
\minitoc

\section{Motivation}

When it comes to decisional process in robotics, two main schools of though can be distinguished: machine learning and deterministic processes such as planing or states machines. Both ways have their advantages and disadvantages. Learning is usually "cheap" (the decision process is quick) and always proposes a solution to a given problem. However, learning requires either a big amount of data or a long period of learning. Moreover, during the learning period, the robot can produce inconsistent behavior which can be confusing for a potential human collaborator. On the other hand, planing can take into account humans through social rules and ensure the validity of a whole solution. However, planing does not learn from human behavior, and, when it comes to complex tasks or environments, it can become slow to propose a solution. The idea of this work is to propose a solution where we combine planing and learning in the context of human-robot interaction in order to take advantage of both. 

This work has been done in collaboration with ISIR at Paris and more particularly with another PhD student Erwan Renaudo. It has been done in the context of the RoboErgoSum ANR project\footnote{http://roboergosum.isir.upmc.fr/}. This work is based on the work of \cite{renaudo2014design} and has been the subject of a publication in a workshop at the RoMan conference \cite{renaudo2015learning} as well as a part of a journal article \cite{khamassi2016integration}.

\section{Background}

\subsection{Inspiration from neurosciences}

Seminal works on living being behaviors have been done in the late 19th century - beginning of the 20th century - with experiments on mammals. One pioneer work concerning the learning process is the experiment of the cat in a box \cite{thorndike1998animal}. In this experiment, a cat is put in a box each time it is hungry. The cat can see food outside of the box and a system of lever allows it to open the box. Each time the cat is put in the box it takes less time to go out. This experiment allows to show the principle of learning through trial and error. 

Latter, studies have highlighted two main kinds of behavior during decision-making tasks. \textbf{Goal-directed behaviors} are governed by estimates of action-outcome contingencies (i.e. decision-making relies on the prior estimation of the outcome expected after an action or an action sequence) and are mainly active at the beginning of the task. Then, when the animal is well trained in the task under stable conditions, a transfer of control to \textbf{habitual behaviors} governed by stimulus-response associations occurs  \cite{dickinson1985actions}. When rodents, monkeys or humans start a new decision-making task, they appear to initially rely on their goal-directed system. They take time to analyse the structure of the task in order to build an internal model of it, and make slow decisions by planning and inferring the long-term consequences of their possible actions before deciding what to do next. Then as their performance gradually improve, they appear to make quicker and quicker decisions, relying on their habitual system which slowly acquires simple stimulus-response associations to solve the task. Finally, when subjects restart to make errors after a task change, they appear to restart planning within their internal model and thus slow down their decision process before acquiring the new task contingencies \cite{balleine2010human, dolan2013goals}. The coordination of these two learning systems allows mammals to avoid long and costly computations when the environment is sufficiently stable, while still enabling animals to detect environmental changes requiring to update their internal model and replan.

In computational neuroscience models, these behaviors are modeled using the theory of Reinforcement Learning \cite{sutton1998introduction}: model-based and model-free algorithms provide a direct analogy with goal-directed and habitual behaviors \cite{daw2005uncertainty}. More recently, different computational criteria have been proposed to decide when to shift between model-based and model-free experts \cite{pezzulo2013mixed, lesaint2014modelling, viejo2015modeling}. Applied to neuroscience tasks, the work from \cite{daw2005uncertainty} proposes that the most certain expert gets control on the agent, while \cite{keramati2011speed} balance speed and accuracy using the cost of planning versus the gain of information. A third approach proposes, in the context of navigation strategies, that a coordination module learns by reinforcement the most efficient behavior (in terms of average obtained reward) in each state \cite{dolle2010path}.

\subsection{Learning in human-robot interaction}

A major part of robotics decision-making algorithms are based on planning processes which take into account a great number of information (\cite{ingrand2014deliberation}). These approaches to decision-making could be seen as similar to what neuroscientists call the goal-directed system, except that there is most of the time no learning in the system. Such approaches have been extended to HRI by taking into account human-aware costs such as social-rules and humans comfort and preferences \cite{cirillo2010human,Lallement2014hatp}.

Besides, robots learning abilities are still very limited and require the injection of important prior knowledge by the human in the robot’s decision-making system. Early applications of reinforcement learning (RL) algorithms to robotics \cite{hayes1994robot, morimoto2001acquisition, smart2002effective} - some of which being neuro-inspired - produced limited progresses, due to applications to relatively simple problems (with a small number of states and actions), to slowness in learning and to systematic instability observed throughout the learning process. More recent applications of RL to robotics have permitted to deal with more complex and continuous action spaces, enabling to learn efficient sensorimotor primitives \cite{kober2011learning, martins2010learning, stulp2013robot}. These approaches have been extended in HRI to allow robots to learn to collaborate with humans.
In several works, the reward signal is interactively assigned by the human \cite{kaplan2002robotic, knox2012reinforcement} while other works use the human to provide demonstrations to the robot \cite{nicolescu2003natural, thomaz2006reinforcement}.
A method of cross-training is used and compared to standard reinforcement learning algorithms in the context of human-robot teamwork in \cite{nikolaidis2013human}. Cross-training is an interactive planning method in which a human and a robot iteratively switch roles to learn a shared plan for a collaborative task. Such approaches to decision-making could be seen as similar to what neuroscientists call habitual behaviors.

Even if we can find more and more interesting works in HRI concerning planing and learning for the robot to collaborate with humans, there is no work to our knowledge concerning how to combine both approaches into a robotics architecture.

\section{Experts presentation}

Inspired from from neuroscience theories and based on the previous work of \cite{renaudo2014design} the aim of this work is to combine goal-directed and habitual behaviors in the context of human-robot Joint Action. To do so, we use two experts which implement these two kinds of behavior. The goal-directed behavior is produced here by HATP (Human-Aware Task Planer), a task planner which has proved its efficiency in the field of human-robot interaction. A Qlearning algorithm allows to implement the habitual behavior. We will describe in this section these two experts and their respective strengths and weakness. The next sections will show how we combined those two experts into two different architectures.

\subsection{HATP}

In our work, the goal-directed behavior is provided by HATP, an HTN (Hierarchical Task Network, \cite{erol1994htn}) task planner which has been conceived to work in the context of human-robot collaboration.  As a HTN planner, HATP uses known preconditions and effects of actions in order to find the best plan that reaches the given goal. It takes as input a list of all possible actions and their description in terms of preconditions and effects and also a description of the current world state as a set of predicates. Then, it looks for the combination of actions that minimizes the solution cost. This cost is computed based on execution time and human-aware costs (e.g the balance of efforts between agents or the waiting time of the human partner). This plan is meant to be executed step by step until the goal is reached. An example of such a plan can be found Fig.~\ref{fig:examplePlan}.

\begin{figure}[!h]
	\centering
    \includegraphics[width=0.9\textwidth]{figs/Chapter7/SharedPlan.png}
    \caption{An example of a Shared Plan computed by HATP. This plan allows a human and a robot to build a stack of colored objects together by placing them one on-top of the others.}
    \label{fig:examplePlan}
\end{figure}

\subsection{Qlearning algorithm (MF)}

The habitual behavior is provided by a model-free reinforcement learning algorithm (MF) that directly learns the state–action associations by caching in each state the earned rewards in the value of each action. In this implementation, the algorithm is implemented as a neural network (see Fig.~\ref{fig:Qlearning}). The network input neurons represent the different possible states and the output neurons encode the estimation of action values in the current state. The weights are modified to associate each state with the most rewarding action in the current task.

A method similar to \cite{brafman2002r} is used to compute the value $Qt(St, a_j)$ of an action $a_j$ in a certain state $S$. $Qt(St, a_j)$ is represented as the scalar product between the input vector and the weights $W_j = (w_{0j} , . . . , w_{Nj})$ linking to this action:
$$Qt(St, a_j) = W_j^t \ . \ (S_t, 1)$$

here we set weights at a positive value to provide an initial optimistic estimate of action values ($w_0$ = 0.5). Weights $W_t$ are updated according to the Qlearning algorithm \cite{watkins1989learning}. The Reward Prediction Error $\delta$ is spread over the weights of the previously active input and the action $a$ done in the corresponding state:
$$\delta = r_t + \gamma_{Hab} \ . \ \underset{b \in A}{max}(W_b^{t-1}) \ . \ S_t - (W_a^{t-1} \ . \ S_{t-1})$$
$$W_a^t = W_a^{t-1} + \alpha_{Hab} \ . \ \delta / \underset{n}{\Sigma}s_{n}$$

with $r_t$ the instant reward received for performing a in $S_{t−1}$ , $\alpha_{Hab}$ the learning rate, $\gamma_{Hab}$ the decay rate of future rewards. The weights are updated locally: only the state from which the action has been performed is updated. Thus, it requires for the agent to visit every known state of the problem to update values.


\begin{figure}[!h]
	\centering
    \includegraphics[width=0.6\textwidth]{figs/Chapter7/Qlearning.png}
    \caption{Habitual expert, modeled as a Qlearning algorithm implemented as a neural network. The expert receives a state $S$ which is projected onto the input neurons $s_i$, defining an input activity. This activity is propagated through the network weights $W$ to generate activity of the action layer. This activity corresponds to the values $Q(S, a_j)$, with each neuron coding for a distinct action. This value distribution is converted in probability distribution using a softmax function, which allows the expert to make a decision $D$ on the next action to perform.}
    \label{fig:Qlearning}
\end{figure}

\subsection{Experts comparison}

The two different experts have really different ways to decide of the next action to execute. Both methods have their advantages and disadvantages:
\begin{itemize}
\item HATP looks for a complete solution to achieve the given goal while the MF only looks for the next action which maximize the probability to get a reward. Consequently, HATP ensures the feasibility of the solution proposed but could find itself in a state where it does not find a valid solution and so where it will not be able to propose an action. In the other hand, the MF does not ensure that its proposed action allows to achieve the goal but will always propose an action to perform.
\item As HATP computes a whole plan to achieve the goal, its cost, in the sense of time to take a decision, is far bigger than the one of the MF which only proposes the next action. However, this difference needs to be weighed by the fact that as an HATP plan is composed of several actions, this cost is not needed at each step of the task. Moreover, this cost stay acceptable in a not so complex task.
\item HATP is conceived to produce a robot behavior understandable and acceptable by the human. The actions it proposes will produce a consistent behavior of the robot with which one the human can easily collaborate. For its part, the MF has a long period of learning during which one the behavior produced is inconsistent and can be really disturbing for a human collaborating with the robot. Moreover, each time a change happens in the task, a new learning phase is needed. However, the MF is able to learn to adapt its behavior to the human whereas the HATP policy is defined off-line and can not be updated with the behavior of the human during the interaction.
\end{itemize}


\section{First architecture: a proof of concept}


\subsection{Control architecture}


\begin{figure}[!h]
	\centering
    \includegraphics[width=0.8\textwidth]{figs/Chapter7/FirstArchi.png}
    \caption{First tried architecture to combine the two experts. The Situation Assessment module gets data from perception and maintains the current world state. This world state is used by the supervision to compute the reward and by the experts to take decisions. The propositions of the two experts are sent to the meta controller which decides of the action to execute. The supervisor executes the action with the help of lower execution modules.}
    \label{fig:FirstArchi}
\end{figure}

The first architecture we tried to combine the two experts is the one Fig.~\ref{fig:FirstArchi}. In this architecture the two experts are placed in parallel. The execution of a task by the architecture follows several steps:
\begin{itemize}
\item The Situation Assessment module receives data from perception and maintains the world state representation. This world state is represented with predicates (see Sec.~\ref{subsec:taskOne}).
\item The supervisor uses the current world state to compute the reward sent to the MF. This reward is a boolean which is true if the current goal is achieved (see Sec.~\ref{subsec:taskOne}). The supervisor also sends to the MF the last tried action (which was not necessarily the one proposed by the MF) in order to update the learning.
\item The experts decide of the next action to execute based on the current world state. The action proposed by the MF for a given world state is sent directly to the meta controller. Concerning HATP, the supervisor monitors the execution of its plan and sends the next action to execute to the meta controller. A new plan is computed by HATP at the beginning of the task or whenever an unexpected situation happens (an action from the plan fails, the human executes an unexpected action or the robot executes an action proposed by the MF which is not in the current plan).
\item Once the proposition of action from each expert is received, the meta controller decides which action the robot should execute. In this first implementation the meta controller uses a random arbitration: the action is chosen with an equal probability for each expert.
\item The supervisor executes the chosen action with the help of lower execution modules (motion planning, control, ...). 
\end{itemize}
These steps are executed one by one until the goal is achieved.



\subsection{Task}

\label{subsec:taskOne}

This first architecture has been tried in a simple task. Moreover, as the learning part of the architecture requires long learning periods, the tests have been done in simulation.

\begin{figure*}[!h]
\centering
	\subfigure[Initial situation]{
        \centering
        \includegraphics[width=0.45\textwidth]{figs/Chapter7/initFirstTask.png}
       \label{subfig:initFirst}
   }
    %~
	\subfigure[Final situation]{
        \centering
        \includegraphics[width=0.47\textwidth]{figs/Chapter7/endFirstTask.png}
       \label{subfig:endFirst}
   }
    \caption{Description of the task used with the first architecture. In this task, the human and the robot have to remove all the objects of the table and put them in the pink box. At the beginning of the interaction two objects are accessible only by the robot and another one only by the human. The box is accessible only by the robot.}
    \label{fig:firstTask}
\end{figure*}

In the chosen task a human and a robot have to "clean a table" together. To do so, they need to remove all the objects from the table and put them in a box (see Fig.~\ref{fig:firstTask}). At the beginning of the interaction two objects are accessible only by the robot and another one only by the human. The box is accessible only by the robot. To achieve the goal, several actions can be executed by the agents:
\begin{itemize}
\item \textbf{Pick an object:} both agents can pick an object accessible by them.
\item \textbf{Throw an object:} the robot can throw an object it has in hand in the box near itself. 
\item \textbf{Give an object:} the robot can give an object to the human.
\item \textbf{Take an object:} the robot can receive an object from the human
\item \textbf{Wait:} the robot can wait for the human to execute an action.
\end{itemize}
All these actions have an impact into the world state. This world state is estimated by the Situation Assessment module and represented with predicates which can be either true or false. For this task, we consider the following predicates:
\begin{itemize}
\item \textbf{<Object, isReachableBy, Agent>:} these predicates represent for each object if it is reachable by the human or the robot.
\item \textbf{<Object, isIn, Box>:} these predicates represent the fact that an object has been thrown in the box.
\item \textbf{<Agent, hasInHand, Object>}: these predicates represent the fact that the human or the robot holds an object.
\end{itemize}
These predicates allow the experts to take their decisions but also the supervisor to compute the reward needed by the MF. The robot will receive a reward whenever all objects are in the box and it performs the \textit{Wait} action. We chose to impose to the robot to perform a \textit{Wait} action at the end of the task in order for it to learn that the task is over and that no more action is needed.

To test our architecture, we compare its performances to the performances of the system running with only the MF and only HATP. We run the experiment in all conditions with a fixed time limit. At the beginning of an experiment the set-up was put at the initial situation (Fig.~\ref{subfig:initFirst}). Once the task is achieved and reward is obtained by the robot, the set-up is put back to the initial situation and the task can be performed again.

As we run the task in simulation, the behavior of the human is also simulated. We chose here to have a collaborative human: it performs all actions HATP planned for him and participates to handover whenever the robot requires one.

\subsection{Results}

\begin{figure*}[!h]
\centering
	\subfigure[Mean cumulative reward on 10 simulations where the robot repeatedly fulfils the task. Errorbars represent the standard deviation from the mean every 100 decisions.]{
        \centering
        \includegraphics[width=0.45\textwidth]{figs/Chapter7/rewardsFirstTask.pdf}
       \label{subfig:rewardsFirstTask}
   }
    %~
	\subfigure[Number of actions tried per experiment. Dashed line is the mean number of action depending on the control method (MF only, HATP only or combination)]{
        \centering
        \includegraphics[width=0.45\textwidth]{figs/Chapter7/DecisionNumberFirstTask.pdf}
       \label{subfig:decisionsFirstTask}
   }
    %~
	\subfigure[Mean MF connection weights evolution for MF alone and MF and HATP combination. The amplitude is defined as the sum of the absolute value of weights. Weights are initialized to zero, thus the higher the amplitude is, the more the MF has learnt which action to do.]{
        \centering
        \includegraphics[width=0.45\textwidth]{figs/Chapter7/WeightEvolFirstTask.pdf}
       \label{subfig:weightsFirstTask}
   }
    \caption{Performance of the developed system compared to systems with only the MF and only HATP. The results are for 10 runs of approximately 30 minutes in each condition.}
    \label{fig:resultsFirstTask}
\end{figure*}

The main criteria used to evaluate our system is the cumulative reward obtained in each run (i.e. the number of time the human and the robot manage to achieve the task in a fixed amount of time). We run 10 times the experiment in each condition (MF only, HATP only and the combination of both) for a duration of approximately 30 minutes. The number of rewards obtained are presented in Fig.~\ref{subfig:rewardsFirstTask}. All experiments last the same fixed time, but the number of decisions taken at the end may vary.  We observe a poor performance of the MF alone, which is not able to solve the task more than three times. As the MF has no initial knowledge, it has to discover the right sequence of actions, which is non trivial with the given number of possible states and actions.
The random combination HATP-MF is performing much better than the MF alone, solving the task 25 times in average. However, HATP alone performs even better solving the task 34 times in average. Indeed, the task is easy enough to solve for HATP and the time required to find a plan is negligible here. As the simulated human always performs the actions planned by HATP, the plan found by HATP is always optimal and will never change during the task execution. Accordingly, the random combination of HATP and the MF performs worst as it can include actions proposed by the MF that make the plan non optimal.

Fig.~\ref{subfig:decisionsFirstTask} shows the number of actions proposed to the supervision system during each experiment. We can see that the MF alone suggests twice to three times more actions than HATP or the combination in the same given time. This is mainly due to the way each Expert decides: the MF only needs to compute the values of each action (which is propagating the state activity to action neurons) and to draw an action from the resulting probability distribution. It proposes a lot of unfeasible actions and the supervision system will not spend time to execute them as it will stop to the preconditions verification. HATP checks for action preconditions when planning and so, for each of the action proposed by HATP, the supervisor spend time to execute it (or try to execute it if the action is not really feasible according to the geometry). The number of actions suggested by the combination of Experts is closer to the one with HATP alone while remaining lightly higher. It can be explained by the fact that a part of the actions proposed by the combination comes from HATP and, for the ones coming from the MF, HATP helps it to learn a solution faster, causing it to propose less unfeasible actions.

Finally, we analyze the effect of combination on learning of MF in Fig.~\ref{subfig:weightsFirstTask}. Learning is evaluated by weights amplitude, namely the sum of weights absolute value over actions. The MF starts with weights initialized to zero, each learning step increases or decreases the value of some of the weights, until convergence. The figure shows that learning occurs much earlier for the combination of Experts than when the MF is alone. The combination has a bootstrapping effect and the knowledge about the task from HATP is transferred to the MF. This shows that a human-provided a priori knowledge can be used to guide exploration and learn quicker. Even if not tested in this experiment, this means that a change in task condition for which HATP can find a new plan can be learnt quickly by the MF, so the robot will be able to adapt to the new conditions without taking too much time.

\subsection{Intermediate conclusion}

The first results obtained with this architecture allow to show that the combination of HATP and the MF allows to bootstrap the MF and to learn faster a policy to achieve the goal.

However, this task is too simple for HATP to be in difficulty when deciding alone. The purpose of the second task and architecture presented in the next section is to show the benefits of the combination of the two experts and more particularly how HATP can benefits from the MF. Moreover, we want to test the reaction of our system to changes in the task as well as a more elaborated arbitration criteria for the meta controller.


\section{Second architecture: the limitations}

\subsection{Control architecture}

\begin{figure}[!h]
	\centering
    \includegraphics[width=0.8\textwidth]{figs/Chapter7/SecondArchi.png}
    \caption{Second tried architecture to combine the two experts. The Situation Assessment module gets data from perception and maintains the current world state. This world state is used by the supervision to compute the reward and by the experts to take decisions. Here the meta controller is placed upstream from the two experts. It first decides which expert should propose an action. Then, the supervisor executes the action of the chosen expert with the help of lower execution modules.}
    \label{fig:SecondArchi}
\end{figure}

One of the advantage of the MF against HATP is its computation time. In the previous architecture, both experts where consulted before the meta controller took a decision. Consequently, even if the MF was chosen, we still lost time to compute plans with HATP. In order to solve this issue, we modified the previous architecture as shown in Fig.~\ref{fig:SecondArchi}. 

In the new architecture, the meta controller is placed upstream from the two experts. Consequently, the order of the previous steps during a task is also slightly modified:
\begin{itemize}
\item The Situation Assessment module still receives data from perception and maintains the world state representation. 
\item The supervisor sends the needed data concerning both experts to the meta controller in order for it to take a decision (see bellow). 
\item Once the meta controller decision taken, we look for the action proposed by the selected expert. If the MF is chosen it directly sends its action to the supervisor as well as data concerning its decision (see bellow). If HATP is chosen, if needed, the supervisor asks for a new plan, else it directly executes the next action of the current plan. A new plan is needed at the beginning of the task or whenever an unexpected situation happen (an action from the plan fails, the human executes an unexpected action or the robot executes an action proposed by the MF which is not in the current plan).
\item The supervisor still executes the chosen action with the help of lower execution modules (motion planning, control, ...). 
\end{itemize}

In this architecture, we also introduced a new arbitration criteria for the meta controller. This criteria is based on the cost of each expert (duration to find a solution) and its prediction error. For the MF, the prediction error is the difference between the probability for the proposed action to lead to a reward and the actual received reward. For HATP, the prediction error is 0 if after the execution of the proposed action the world state corresponds to what HATP predicted (based on the action effects) and 1 if it differs.

$$P^E_t = \alpha \ . \ err^E_t + \beta \ . \ cost^E_t$$
with $P^E_t$ the probability for the expert $E$ to be chosen by the meta controller at a time $t$, $err^E_t$ the prediction error of the expert $E$ at a time $t$ and $cost^E_t$ the cost of an expert $E$ at a time $t$. $\alpha$ and $\beta$ are parameters. The prediction error and the cost of the experts are averaged through time:
$$err^E_t = (1- \gamma_{err}) \ . \ err^E_{t-1} + \gamma_{err} \ . \ err^E_t$$
$$cost^E_t = (1- \gamma_{cost}) \ . \ cost^E_{t-1} + \gamma_{cost} \ . \ cost^E_t$$
with $\gamma_{err}$ and $\gamma_{cost}$ parameters.

\subsection{Task}

The previous task was too simple to have difficulties with HATP as the only expert. The new task is an upgrade of the previous one with several additions.


\paragraph{More complex task}

A first way to complexify the task for HATP is to increase the combinatory of the task. Indeed, there was not too much ways to solve the previous task, so, HATP didn't need too much time to compute a plan. The goal of the new task is still to "clean a 
table", however, there are now two different boxes where to put the objects. The blue objects have to go in the blue box and the green objects have to go in the green box. We increased the number of objects in the task: at the beginning of the interaction 6 objects (3 blue and 3 green) are randomly placed on 7 possible placement in the table (see Fig.~\ref{subfig:InitSecondScenario1} and Fig.~\ref{subfig:InitSecondScenario2}). 

We also add some new possible actions for the robot:
\begin{itemize}
\item \textbf{Pick an object:} both agents can still pick the objects accessible by them.
\item \textbf{Throw an object:} the robot can throw an object it has in hand in a box of the same color accessible by itself. The human can throw an object it has in hand in the blue box. 
\item \textbf{Give an object:} the robot can still give an object to the human.
\item \textbf{Take an object:} the robot can still receive an object from the human
\item \textbf{Place an object on a placement:} the robot can place an object it has in hand on a placement accessible by itself.
\item \textbf{Navigate to another position:} the robot can navigate to another position in order to change the objects it can reach. The two possible positions for the robot are the one in Fig.~\ref{subfig:InitSecondScenario1} and Fig.~\ref{subfig:InitSecondScenario2}) and the one in Fig.~\ref{subfig:solutionSecondScenario}.
\item \textbf{Wait:} the robot can still wait for the human to execute an action.
\end{itemize}
The predicates used to represent the world state also changed. They are now composed of:
\begin{itemize}
\item \textbf{<Object, isReachableBy, Agent>:} these predicates represent for each objects if they are reachable by the human or the robot.
\item \textbf{<Placement, isReachableBy, Agent>:} these predicates represent for each placement if they are reachable by the human or the robot.
\item \textbf{<Box, isReachableBy, Agent>:} these predicates represent for each box if they are reachable by the human or the robot.
\item \textbf{<Object, isIn, Box>:} these predicates represent the fact that an object has been throw in a box.
\item \textbf{<Agent, hasInHand, Object>}: these predicates represent the fact that the human or the robot holds an object.
\item \textbf{<Object, isOn, Placement>}: these predicates represent the fact that an object is on a specific placement.
\item \textbf{<Robot, isAt, Position>}: these predicates represent the position of the robot (Position are the two possible places it can navigate to).
\end{itemize}
In this task, a reward is given to the robot whenever all objects are in a box.

\begin{figure*}[!h]
\centering
	\subfigure[One possible initial set-up. In this situation the robot can access four objects (two blue and two green) as well as the green box.]{
        \centering
        \includegraphics[width=0.3\textwidth]{figs/Chapter7/InitSecondScenario1.png}
       \label{subfig:InitSecondScenario1}
   }
    %~
	\subfigure[Another possible initial set-up. In this situation the robot thinks it can access the blue object in the middle of the table. However, the green object in front of it blocks its access.]{
        \centering
        \includegraphics[width=0.3\textwidth]{figs/Chapter7/InitSecondScenario2.png}
       \label{subfig:InitSecondScenario2}
   }
    %~
	\subfigure[One possible way for the robot to access the blue object it was not able to reach is to move to another position.]{
        \centering
        \includegraphics[width=0.3\textwidth]{figs/Chapter7/solutionSecondScenario.png}
       \label{subfig:solutionSecondScenario}
   }
    \caption{Description of the task used with the second architecture. In this task, the human and the robot have to remove all the objects of the table and put them in the box of the same color. At the beginning of the interaction several objects are accessible by the robot, others by the human and others by both agents. The green box is accessible by the robot and the blue one by the human. The placements are the white squares on the table.}
    \label{fig:firstTask}
\end{figure*}

As there are more objects and more actions to perform for the robot, the number of possible ways to achieve the task highly increases. Indeed, in the initial set-up of the previous task HATP needed around 145ms to find a plan to achieve the goal. In this new task it takes around 22s to find a plan. Consequently, the difference of cost between the MF and HATP should make more difference in this experiment.

\paragraph{Difference between planning and geometry}

In order to get closer from possible real life situation, we introduced a geometrical problem in the task. Indeed, sometimes it can happen that the knowledge computed by the robot is not accurate and that, consequently, the computed plan is not valid at execution. In our task, there are two placements in the middle of the table (accessible both by the human and the robot) which are close to each other. Each time an object is on one of these placement, the robot thinks it can reach it. when there is an object in only one of the placement (as in Fig.~\ref{subfig:InitSecondScenario1}) the robot can effectively reach the object. However, when there is an object in both placements, the robot cannot reach the one in the farthest placement as the other one blocks its access (see Fig.~\ref{subfig:InitSecondScenario2}). The Situation Assessment is not able to differentiate the two situations and in each case it will estimate that all objects in these placements are reachable by the robot. The robot will discover that it can not reach an object at motion planning time and so, the initial HATP plan will not take this into account (but when the action to pick the object not reachable failed, the robot will update its knowledge and so the new HATP plan). To access an object not reachable by it the robot can either navigate to another position (as in Fig.~\ref{subfig:solutionSecondScenario}), remove the object which blocks the access or get the object from the human (through handover).

The MF should allow the robot to learn in which case an object is really reachable by the robot and in which case another solution is preferred to get the object.

\paragraph{Different human behaviors}

Finally, in the previous task, one of the reason HATP was performing very well was that the human always executed the actions planned for him. In real life, even if the human is collaborative, he does not necessarily take the same decisions as the ones HATP took for him. In this task, we introduced three different kinds of human behavior:
\begin{itemize}
\item \textbf{The collaborative human:} it picks all objects accessible by him (with a priority for blue ones), throws the blue objects in the blue box and participates to all handover engaged by the robot.
\item \textbf{The anti-handover human:} it picks all the blue objects accessible by him, throws the blue objects in the blue box but does not participate to handover engaged by the robot
\item \textbf{The lazy human:} it picks the blue objects accessible only by him (and not the ones the robot can access), throws the blue objects in the blue box and does not participate to handover engaged by the robot
\end{itemize}
COMMENTAIRE: Que fait le robot quand l'humain refuse les handover?
\subsection{Results}

\begin{figure*}[!h]
\centering
	\subfigure[Mean cumulative reward for the system with only HATP. We can see that the system performs better with a collaborative human (nice), then with an human rejecting handover (noHand) and then with a lazy human (lazy).]{
        \centering
        \includegraphics[width=0.45\textwidth]{figs/Chapter7/reward_HATP.pdf}
       \label{subfig:reward_HATP}
   }
    %~
	\subfigure[Mean cumulative reward for the system with only the MF. Different values for the MF parameters have been tested in order to find the best configuration in this task]{
        \centering
        \includegraphics[width=0.45\textwidth]{figs/Chapter7/reward_MFalone.pdf}
       \label{subfig:reward_MFalone}
   }
    %~
	\subfigure[Mean cumulative reward for the system with the combination of both experts; Different values for the arbitration criteria parameters have been tested in order to find the best configuration in this task.]{
        \centering
        \includegraphics[width=0.45\textwidth]{figs/Chapter7/reward_combo.pdf}
       \label{subfig:reward_combo}
   }
    \caption{Mean cumulative reward for each conditions tested (HATP only, MF only and combination). The results are for 10 runs of approximatively 40 minutes in each conditions where the robot repeatedly fulfils the task.}
    \label{fig:resultsFirstTask}
\end{figure*}


We first tried the new architecture and task with HATP as the only expert. We can see in Fig.\ref{subfig:reward_HATP} that, as expected, HATP performs better with a collaborative human which will have a behavior closer than the one its planned that with humans with less collaborative behaviors. 

Then, we tested the system with the MF alone and with different parameters of the learning algorithm in order to get the better possible instantiation for this task. In a first step, we did it with a collaborative human and with only one possible initial set-up without geometrical complications. We can see in Fig.~\ref{subfig:reward_MFalone} that, as expected, the MF alone performs poorly compared to HATP.

Then, we tested the combination of both experts. In a first step, we tested it with the collaborative human and with only one possible initial set-up without geometrical complications. We tested several parametrisations of the arbitration criteria in order to get the best implementation for this task. However, we noticed that, even when we put the system in the best possible situation, it performs barely as well as HATP in its worst case (the task was solved around 45 times in each cases, see Fig.~\ref{subfig:reward_combo}). Indeed, with a more complex task, the bootstrap effect of HATP was not enough for the MF to learn a sufficiently good action policy. We tried with some runs way longer (several hours) but it was still not sufficient for the MF to learn a correct policy. 

Moreover, even if we tried to reduce computation time by putting the meta controller upstream from the experts, the effect was not the one expected. Indeed, the meta controller here is probabilist and so, even if the probability to choose the MF becomes higher than HATP, it can still happen for HATP to be chosen. In this case, a whole plan is computed by HATP even if we ask it only one action during the task. Consequently, the planning time remains the same than if HATP follows its plan alone to achieve the task.

\section{Conclusion}

In this chapter, we presented an architecture allowing to combine learning (a model free algorithm) and planing (a human-aware task planer HATP) during the robot decisional process. First results shown that HATP allows to bootstrap the learning and so to quickly learn a consistent and acceptable behavior for the robot.

In a second time, we tried to show the benefits of the learning in the system. Despite the facts that the result was not the one expected, we can still learn some lessons from this work and think of solutions to improve the system. One first possible modification would be to rework on the learning algorithm in order to study if there is method more adapted to this context. Then, another amelioration would be to look for a new arbitration criteria between the two experts. Maybe a criteria with an hysteresis in order to reduce switches between experts in a task and allow them to have time to develop their own strategy (and not having one expert breaking the strategy the other tried to set-up) would be a good idea. Finally, one interesting idea is to allow HATP to have a feedback on what is learned by the MF. Indeed, the knowledge of HATP concerning the actions is put off-line and is not updated during the interaction. For example, maybe the learning can provide the real time needed to execute an action or its probability of success given what was learned from previous interactions.


\ifdefined\included
\else
\bibliographystyle{StyleThese}
\bibliography{These}
\end{document}
\fi



\ifdefined\included
\else
\documentclass[english, a4paper,11pt,twoside]{StyleThese}
\include{formatAndDefs}
\sloppy
\begin{document}
\fi


\chapter*{Conclusion}
\addstarredchapter{Conclusion} %Sinon cela n'apparait pas dans la table des matières


\section*{Contributions}

\section*{Possible ameliorations}

\begin{itemize}
\item improve of thesis works
\item work on engagement
\end{itemize}

\ifdefined\included
\else
\bibliographystyle{StyleThese}
\bibliography{These}
\end{document}
\fi


\appendix

\chapter{Terms of the formalization}
\label{chap:annexe1}


\hspace{18pt} \textbf{$TS$:} state of the task from the robot point of view.
$$TS = <g_R, SP, WS>$$

\textbf{$MS(H)$:} mental state of the human $H$.
$$MS(H) = <g_H, g_R(H), SP(H), WS(H)>$$

\bigskip
\textbf{Goals:}

\textbf{$g_R$:} current goal of the robot

\textbf{$g_H$:} current goal of the human $H$

\textbf{$g_R(H)$:} current goal of the robot from the human $H$ point of view

$$g = <Name_g, Actors_g, Params_g, Obj_g>$$
\indent \textbf{$Name_g$:} identifier of the goal $g$

\textbf{$Actors_g$:} actors of the goal $g$

\textbf{$Params_g$:} parameters of the goal $g$

\textbf{$Obj_g$:} objectives of the goal $g$

\textbf{$label_g$:} state of a goal $g$ already over (DONE or ABORTED)

\bigskip
\textbf{Shared Plan:}

\textbf{$SP$:} current Shared Plan

\textbf{$SP(H)$:} current Shared Plan from the human $H$ point of view
$$SP = <id_p, A_p, L_p>$$

\textbf{$id_p$:} identifier of the plan $p$

\textbf{$A_p$:} actions of the plan $p$
$$A_p = <A_{prev}, A_{cur}, A_{next}, A_{later}>$$

\textbf{$L_p$:} links between actions for the plan $p$
 $$l \in L_p = \langle prev_l, \ next_l \rangle$$

\textbf{$prev_l$:} identifier of the action to execute first

\textbf{$next_l$:} identifier of the action to execute next (after $prev_l$)

\bigskip
\textbf{Actions sets:}

\textbf{$A_{prev}$:} actions already executed

\textbf{$A_{prev}(H)$:} actions already executed from the human $H$ point of view

\textbf{$A_{prev}^R$:} actions already executed by the robot

\textbf{$A_{prev}^R(H)$:} actions already executed by the robot from the human $H$ point of view

\textbf{$A_{prev}^H$:} actions already executed by the human $H$

\textbf{$A_{prev}^H(H)$:} actions already executed by the human $H$ from the human $H$ point of view

\bigskip
\textbf{$A_{cur}$:} actions in progress

\textbf{$A_{cur}(H)$:} actions in progress from the human $H$ point of view

\textbf{$A_{cur}^R$:} actions in progress and executed by the robot

\textbf{$A_{cur}^R(H)$:} actions in progress and executed by the robot from the human $H$ point of view

\textbf{$A_{cur}^H$:} actions in progress and executed by the human $H$

\textbf{$A_{cur}^H(H)$:} actions in progress and executed by the human $H$ from the human $H$ point of view

\bigskip
\textbf{$A_{next}$:} actions from the plan which need to be performed

\textbf{$A_{next}(H)$:} actions from the plan which need to be performed from the human $H$ point of view

\textbf{$A_{next}^R$:} actions from the plan which need to be performed by the robot

\textbf{$A_{next}^R(H)$:} actions from the plan which need to be performed by the robot from the human $H$ point of view

\textbf{$A_{next}^H$:} actions from the plan which need to be performed by the human $H$

\textbf{$A_{next}^H(H)$:} actions from the plan which need to be performed by the human $H$ from the human $H$ point of view

\textbf{$A_{next}^X$:} actions from the plan which need to be performed and are not yet allocated

\textbf{$A_{next}^X(H)$:} actions from the plan which need to be performed and are not yet allocated from the human $H$ point of view

\bigskip
\textbf{$A_{later}$:} actions from the plan which need to be performed later

\textbf{$A_{later}(H)$:} actions from the plan which need to be performed later from the human $H$ point of view

\textbf{$A_{later}^R$:} actions from the plan which need to be performed later by the robot

\textbf{$A_{later}^R(H)$:} actions from the plan which need to be performed later by the robot from the human $H$ point of view

\textbf{$A_{later}^H$:} actions from the plan which need to be performed later by the human $H$

\textbf{$A_{later}^H(H)$:} actions from the plan which need to be performed later by the human $H$ from the human $H$ point of view

\textbf{$A_{later}^X$:} actions from the plan which need to be performed later and are not yet allocated

\textbf{$A_{later}^X(H)$:} actions from the plan which need to be performed later and are not yet allocated from the human $H$ point of view

\bigskip
\textbf{Action:}
$$a = < id_{a}, \ Name_{a}, \ Ag_{a}, \ Params_{a}, \ Precs_{a}, \ Effects_{a} >$$

\textbf{$id_{a}$:} identifier of the action $a$

\textbf{$Name_{a}$:} name of the action $a$

\textbf{$Ag_{a}$:} actors of the action $a$

\textbf{$Params_{a}$:} parameters of the action $a$

\textbf{$Precs_{a}$:} preconditions of the action $a$

\textbf{$Effects_{a}$:} effects of the action $a$

\textbf{$label_{a}$:} state of an action $a$ already executed (DONE, FAILED or ABORTED)

\bigskip
\textbf{World State:}

\textbf{$WS$:} current world state from the robot point of view

\textbf{$WS(H)$:} current world state from the human $H$ point of view

$$p \in WS = <entity, attribute, value>$$


\chapter{Questionnaire of the on-line video based study for the robot head behavior}
\label{chap:annexe2}

\section{Anticipation of robot actions}

\begin{figure}[!h]
	\centering
    \includegraphics[width=0.7\textwidth]{figs/Chapter6/QuestionsSce1.png}
    \caption{Questions asked to participant for each videos of the scenario concerning the anticipation of robot actions.}
    \label{fig:headArchi}
\end{figure}

\begin{figure}[!h]
	\centering
    \includegraphics[width=0.7\textwidth]{figs/Chapter6/ComparaisonSce1.png}
    \caption{Question asked to participant to compare the two videos they watched of the scenario concerning the anticipation of robot actions.}
    \label{fig:headArchi}
\end{figure}

\section{Following human's activity}


\begin{figure}[!h]
	\centering
    \includegraphics[width=0.7\textwidth]{figs/Chapter6/QuestionsSce2.png}
    \caption{Questions asked to participant for each videos of the scenarios concerning the following of human's activity.}
    \label{fig:headArchi}
\end{figure}

\subsection{Signalling human's actions}

\begin{figure}[!h]
	\centering
    \includegraphics[width=0.7\textwidth]{figs/Chapter6/QuestionsSce4.png}
    \caption{Questions asked to participant for each videos of the scenarios concerning the signalling of human's actions.}
    \label{fig:headArchi}
\end{figure}


\subsection{Finding the priority target}

\begin{figure}[!h]
	\centering
    \includegraphics[width=0.7\textwidth]{figs/Chapter6/QuestionsSce6.png}
    \caption{Questions asked to participant for each videos of the scenarios concerning the signalling of human's actions.}
    \label{fig:headArchi}
\end{figure}

\chapter{Questionnaires of the user-study}
\label{chap:annexe3}

\section{Reminder questionnaire}


\begin{figure}[!h]
	\centering
    \includegraphics[width=\textwidth]{figs/Chapter5/Questionnaire_rappel_english.png}
    \caption{}
    \label{fig:Questionnaire_rappel_english}
\end{figure}

\newpage
\section{General questionnaire}

\begin{figure}[!h]
	\centering
    \includegraphics[width=\textwidth]{figs/Chapter5/Questionnaire_english1.pdf}
    \caption{}
    \label{fig:Questionnaire_english}
\end{figure}

\begin{figure}[!h]
	\centering
    \includegraphics[width=\textwidth]{figs/Chapter5/Questionnaire_english2.pdf}
    \caption{}
    \label{fig:Questionnaire_english}
\end{figure}


\ifdefined\included
\else
\documentclass[english,a4paper,11pt,twoside]{StyleThese}
\include{formatAndDefs}
\sloppy
\begin{document}
\setcounter{chapter}{0} %% Numéro du chapitre précédent ;)
\dominitoc
\faketableofcontents
\fi

\chapter{French extended abstract}

\label{ch:resume}

\selectlanguage{francais}
\section{Introduction}

Dans les années 40, des chercheurs inventent les premières machines appelées ordinateurs. En 1956, à la conférence de Darmouth, le domaine de l'intelligence artificielle est reconnu et les premiers robots arrivent rapidement dans notre environnement. Certains de ces robots vont devoir évoluer avec les Hommes ou dans leur entourage. Entre autres, les robots "co-workers" en industrie ou les robots sociaux \cite{dautenhahn2007socially}. Le but de cette thèse est de se rapprocher de robots qui peuvent agir conjointement avec les Hommes de manière naturelle, fluide et efficace. On se concentre ici sur les problématiques liées aux processus décisionnels durant l'action conjointe Homme-Robot. 

Dans un premier temps, basé sur une étude bibliographique des éléments nécessaires à l'action conjointe entre Hommes ainsi que sur des travaux existants en interaction Homme-Robot, les différents éléments nécessaires à l'action conjointe Homme-Robot seront identifiés ainsi que la manière dont ils peuvent s'articuler dans une architecture. Puis, l'architecture du superviseur, contribution technique principale de la thèse, sera présentée. Dans un second temps, mes travaux concernant l'amélioration de la gestion des plans partagés par le robot durant l'action conjointe seront présentés. La première amélioration concerne la prise en compte des états mentaux des Hommes durant l'exécution de plans partagés. La seconde contribution concerne le report de certaines décisions prises initialement par le robot durant l'élaboration du plan et à l'exécution afin d'obtenir une gestion plus flexible des plans partagés. L'évaluation de ces deux contributions en simulation et à l'aide d'une étude utilisateur sera également présentée. Finalement, dans un troisième temps, deux autres contributions à l'action conjointe Homme-Robot seront présentées. La première concerne la gestion du comportement non-verbal et plus précisément de la tête du robot. La seconde concerne l'association d'un système d'apprentissage à un système de planification dans le cadre de la prise de décision haut niveau.


\section{De l'action conjointe entre Hommes à la supervision pour l'interaction Homme-Robot}

\subsection{De l'action conjointe entre Hommes à l'action conjointe Homme-Robot}


\subsubsection{Théorie de l'action conjointe}

L'action conjointe a été décrite par \cite{sebanz2006joint} comme:

\begin{quote}
\textit{n'importe quelle forme d'interaction sociale où deux individus ou plus coordonnent leurs actions dans l'espace et le temps pour apporter un changement dans l'environnement.}
\end{quote}

Plusieurs prérequis sont nécessaires pour que deux individus réalisent avec succès une action conjointe. 

La première chose requise est que ces individus partagent un but et l'intention d'achever ce but. \cite{tomasello2005understanding} défini un but comme la représentation d'un état désiré par un agent et une intention comme un plan d'action qu'un agent s'engage à exécuter pour réaliser le but (basé sur le travail de Bratman \cite{bratman1989intention}). Dans le cas de l'action conjointe, une des définitions les plus reconnues est celle de Bratman \cite{bratman1993shared} qui présente trois conditions pour que deux individus partagent une \textit{intention jointe} d'accomplir un but G :
\begin{enumerate}
\item Chaque individu a l'intention d'accomplir G.
\item Chaque individu a cette intention en accord avec 1 et les parties du plan partagé de 1.
\item 1 et 2 sont une connaissance commune entre chaque individu.
\end{enumerate}
Cette définition est reprise et illustrée par Tomasello et al. dans \cite{tomasello2005understanding} (Fig.~\ref{fig:intention_jointe}). Ils définissent un \textit{but partagé} comme une représentation d'un état désiré plus la connaissance que le but va être réalisé en collaboration et une \textit{intention jointe} comme un plan partagé auquel les agents se sont engagés pour réaliser le but. Concernant ce plan partagé, cette notion a été introduite et formalisée par Grosz and Sidner \cite{grosz1988plans}. Leur définition suggère que chaque agent ne connaît pas nécessairement le plan entier mais seulement la partie qui le concerne et les parties en intersection avec celles de ses partenaires.

\begin{figure}[!h]
	\centering
    \includegraphics[width=\textwidth]{figs/Chapter1/intention_jointe.png}
    \caption{Exemple d'une activité collaborative par Tomasello et al. Ici les deux hommes ont pour \textbf{but partagé} d'ouvrir la boite ensemble. Ils ont choisi un moyen d'atteindre ce but qui prend en compte les capacités de chaque agent et ont donc une \textit{intention jointe}.}
    \label{fig:intention_jointe}
\end{figure}

Un deuxième prérequis de l'action jointe est que chaque agent doit être capable de percevoir et de prédire les actions de ses partenaires et leurs effets. A partir des travaux de \cite{sebanz2006joint}, \cite{pacherie2011phenomenology} et \cite{obhi2011moving} nous avons identifié plusieurs capacités nécessaires à ces prédictions :
\begin{itemize}
\item \textbf{L'attention jointe :} la capacité d'un agent à diriger son attention vers le même objet que ses partenaires de manière à partager la même représentation de l'environnement et des événements.
\item \textbf{Observation de l'action :} plusieurs études ont montré que quand quelqu'un observe une autre personne réaliser une action, une représentation de cette action est formée par l'observateur par ce qu'on appelle les \textit{neurones miroirs} et permet de prédire les effets de l'action \cite{rizzolatti2004mirror}.
\item \textbf{Co-représentation :} avoir une représentation de son partenaire (son but, ses capacités, ses connaissances, etc...) permet de prédire ses actions futures.
\item \textbf{Agency :} la capacité d'attribuer les effets d'une action au bon acteur.
\end{itemize}
Grâce à ces capacités, plusieurs prédictions peuvent être effectuées :
\begin{itemize}
\item \textbf{Quoi :} prédire quelle action un agent va réaliser.
\item \textbf{Quand :} prédire quand une action va avoir lieu et combien de temps elle va durer pour mieux se coordonner dans le temps.
\item \textbf{Où :} prédire les futures positions de ses partenaires pour mieux se coordonner dans l'espace.
\end{itemize}

Finalement, pendant l'action conjointe, les agents doivent être capables de coordonner leurs actions dans le temps et l'espace. Deux sortes de coordination sont définies dans \cite{knoblich20113} :
\begin{itemize}
\item \textbf{La coordination émergente :} qui a lieu intentionnellement. La coordination émergente peut être due à plusieurs mécanismes tels que \textit{l'entraînement} \cite{richardson2007rocking}, des \textit{affordances} communes \cite{gibson1977perceiving} ou la perception d'une action.
\item \textbf{La coordination planifiée :} qui est, elle, intentionnelle. Pour cela, les agents peuvent modifier leur comportement avec ce qui est défini par \cite{vesper2010minimal} comme des \textit{coordination smoothers} (mouvements plus prédictibles, signaux de coordination, etc...) ou utiliser la communication verbale ou non-verbale \cite{clark1996using}.
\end{itemize}


\subsubsection{Comment doter un robot des capacités nécessaires à l'action conjointe ?}

Plusieurs travaux ont déjà été réalisés afin de doter le robot des capacités nécessaires à la réalisation d'une action conjointe avec l'Homme. 

Dans un premier temps, le robot doit être capable de s'engager dans une action conjointe, de choisir un but. Ce but peut être imposé par l'utilisateur mais le robot doit aussi être capable de pro-activement proposer son aide. Pour faire cela, plusieurs travaux ont été réalisés concernant la reconnaissance de plans \cite{ramirez2009plan, bui2003general, singla2011abductive} et d'intentions \cite{breazeal2009embodied, baker2014modeling}. Le robot doit également être capable de choisir entre différents buts possibles. Pour faire cela le domaine de l'intelligence artificielle a commencé à proposer plusieurs solutions \cite{ghallab1994representation, lemai2004interleaving, roberts2016goal}. Une fois le robot engagé dans une action conjointe, il doit être capable de surveiller l'engagement de ses partenaires. Des réponses à ce problème ont été données en utilisant les signaux visuels et gestes \cite{sanghvi2011automatic} ou le contexte et les états mentaux \cite{salam2015multi}. Enfin, une fois le robot engagé dans un but, il doit être capable d'obtenir un plan partagé. Ce plan peut être imposé par l'utilisateur et le robot doit alors être capable de le comprendre \cite{pointeau2014successive, Mohseni2015} et éventuellement de le retransmettre \cite{petit2013coordinating, sorce2015proof}. Le plan peut aussi être construit en collaboration \cite{allen2002human} ou élaboré par le robot \cite{cirillo2010human,Lallement2014hatp}. Si le robot élabore le plan, il doit également être capable de communiquer à son sujet \cite{milliez2016using}.

Afin de mieux communiquer et travailler avec l'Homme, le robot doit être capable d'aligner sa représentation du monde (données en \textit{x, y, z} venant des capteurs) avec celle de l'Homme (relations sémantiques entre objets). Ce processus a été étudié et s’appelle \textit{l'ancrage} \cite{coradeschi2003introduction, mavridis2005grounded, lemaignan2012grounding}. Le robot doit également être capable de représenter son environnement non seulement de son point de vue mais aussi de celui de ses partenaires. La prise de perspective du robot \cite{breazeal2006using,milliez2014framework} peut être utilisée pour résoudre des situations ambiguës \cite{ros2010one}, mieux interagir durant le dialogue \cite{ferreira2015users} ou reconnaître et interpréter les actions de l'Homme \cite{baker2014modeling, nagai2015probabilistic}.

Finalement, le robot doit être capable de se coordonner avec l'Homme. A un haut niveau, l'Homme et le robot doivent coordonner leurs actions afin de réaliser le plan partagé avec succès. Plusieurs systèmes permettent de faire cela tels que \textit{Chaski} \cite{shah2011improved}, \textit{Pike} \cite{karpas2015robust} ou \textit{SHARY} \cite{clodic2009shary}. Pour faire cela, le robot doit se reposer sur ses capacités de communication verbale \cite{roy2000spoken, lucignano2013dialogue, ferreira2015users} et non-verbale \cite{breazeal2005effects, boucher2010facilitative, mutlu2009footing, hart2014gesture}. A un plus bas niveau, le robot doit se coordonner avec l'Homme durant l'exécution d'actions telles que le transfert d'un objet. Cela représente plusieurs challenges tels que trouver des postures acceptables par l'Homme \cite{cakmak2011human, dehais2011physiological, mainprice2012sharing}, approcher un Homme \cite{walters2007robotic} ou produire des trajectoires lisibles et prédictibles \cite{sisbot2012human, kruse2013human}.


\subsubsection{Une architecture trois niveaux}

Nous avons vu précédemment les prérequis pour l'action conjointe entre Hommes et Homme-Robot. Nous allons maintenant voir comment ces éléments se combinent en une architecture trois niveaux.

\begin{figure}[!h]
	\centering
    \includegraphics[width=0.8\textwidth]{figs/Chapter1/Pacherie.png}
    \caption{Les niveaux de Pacherie mis en cascade. Chaque niveau contrôle l'action à un niveau différent.}
    \label{fig:Pacherie}
\end{figure} 

En ce qui concerne l'action conjointe entre Hommes, Pacherie \cite{pacherie2011phenomenology} défend le fait que les processus liés à l'action conjointe se décomposent en trois niveaux qui ont chacun leur rôle  (Fig.~\ref{fig:Pacherie}) :
\begin{itemize}
\item \textbf{Shared Distal Intention :} c'est le niveau le plus haut. Ce niveau est responsable de la formation d'une \textit{intention jointe} et de la gestion du plan partagé.
\item \textbf{Shared Proximal Intention :} ce niveau a la responsabilité d'ancrer les actions reçues du niveau supérieur dans le contexte actuel. Cela doit être fait de manière coordonnée avec les partenaires de l'action conjointe.
\item \textbf{Coupled Motor Intention :} c'est le niveau le plus bas. Il est responsable des commandes moteurs des agents. Il s'occupe de la coordination spatio-temporelle au niveau le plus précis.
\end{itemize}

10 années avant que Pacherie développe ses idées concernant l'architecture trois niveaux, le domaine de la robotique autonome concevait intuitivement des architectures avec trois niveaux très similaires comme dans \cite{alami1998architecture} où l'on retrouve les niveaux :
\begin{itemize}
\item \textbf{Niveau décisionnel :} il est responsable de la production et la supervision du plan d'action. Il peut être comparé au niveau \textit{Distal} de Pacherie.
\item \textbf{Niveau exécutionnel :} il a la responsabilité de choisir, paramétrer et synchroniser les différentes fonctions nécessaires à l’exécution des actions venant du niveau décisionnel. Il peut être comparé au niveau \textit{Proximal} de Pacherie.
\item \textbf{Niveau fonctionnel :} il comprend toutes les foncions bas niveau d'action et de perception du robot. Il peut être comparé au niveau \textit{Motor} de Pacherie.
\end{itemize}

Cette architecture a été développée et adaptée au domaine de l'interaction Homme-robot. Récemment, nous avons présenté dans \cite{devin2016some} une version théorique d'une architecture adaptée à l'action conjointe Homme-robot (Fig.~\ref{fig:ArchiThreeLevels}).

\begin{figure}[!h]
	\centering
    \includegraphics[width=\textwidth]{figs/Chapter1/architecture.png}
    \caption{Architecture récente pour l'action conjointe Homme-robot. L'architecture est organisée autour des trois niveaux définis par Pacherie.}
    \label{fig:ArchiThreeLevels}
\end{figure}

\newpage
\subsection{Supervision pour l'interaction Homme-Robot}


\subsubsection{Rôle du superviseur dans l'architecture globale}

Le superviseur faisant l'objet de cette thèse fait partie d'une architecture globale pour l'interaction Homme-Robot développée au LAAS-CNRS. Une version simplifiée de cette architecture peut être trouvée Fig.~\ref{fig:GlobalArchi}. 


\begin{figure}[!h]
	\centering
    \includegraphics[width=0.7\textwidth]{figs/Chapter2/archiGlobal.png}
    \caption{Architecture globale pour l'interaction Homme-robot développée au LAAS-CNRS.}
    \label{fig:GlobalArchi}
\end{figure}

Cette architecture est composée de :
\begin{itemize}
\item \textbf{Un niveau sensorimoteur :} qui contient les modules bas niveau du robot lui permettant de gérer ses capteurs et actionneurs.
\item \textbf{TOASTER :} un module permettant au robot de représenter et maintenir un état du monde symbolique de son point de vue ainsi que de celui de ses partenaires.
\item \textbf{GTP :} un planificateur géométrique permettant au robot d'effectuer des actions en prenant en compte le confort et la sécurité de 
l'Homme.
\item \textbf{HATP :} un planificateur symbolique permettant au robot de calculer des plans pour lui même et pour ses partenaires afin d'atteindre un but donné.
\item \textbf{Un module de dialogue :} permettant au robot de communiquer avec l'Homme.
\item \textbf{Un superviseur :} ayant la charge de superviser l'activité du robot en coordonnant les autres modules. Il choisit le but du robot, veille au bon déroulement du plan partagé, choisit quand exécuter une action et comment communiquer.
\end{itemize}

L'architecture interne du superviseur peut être trouvée Fig.~\ref{fig:archiSup}. Il est composé de plusieurs modules :
\begin{itemize}
\item \textbf{Un gestionnaire de but :} permettant au robot de choisir quel but exécuter à chaque moment.
\item \textbf{Un élaborateur de plan :} permettant au superviseur de communiquer avec HATP afin d'obtenir un plan partagé.
\item \textbf{Un mainteneur de plan :} permettant au robot de suivre l'évolution du plan partagé.
\item \textbf{Un estimateur d'états mentaux :} permettant au robot d'estimer les états mentaux de ses partenaires humains concernant le plan partagé.
\item \textbf{Un module de décision :} permettant au robot de choisir quand exécuter une action ou donner une information.
\item \textbf{Un module d'exécution d'action :} permettant au superviseur de surveiller la bonne exécution des actions par le robot.
\item \textbf{Un module de communication non-verbale :} permettant de gérer pour le moment uniquement la tête du robot.
\end{itemize}

\begin{figure}[!h]
	\centering
    \includegraphics[width=\textwidth]{figs/Chapter2/ArchiSup.png}
    \caption{Architecture interne du superviseur. Les modules en gras sont traités dans ce manuscrit.}
    \label{fig:archiSup}
\end{figure}

\newpage
\section{Les plans partagés durant l'action conjointe Homme-Robot}

\subsection{Prendre en compte les états mentaux pendant l'exécution de plans partagés}

\subsubsection{Motivations et précédents travaux}

Quand le robot interagit avec un Homme, il est important qu'il ne le considère pas comme un outil ou un obstacle mais qu'il prenne en compte ses sentiments et son confort et donc son point de vue notamment lors de l'exécution d'un plan partagé.

La théorie de l'esprit désigne la capacité qu'ont les humains de reconnaître et s'attribuer des états mentaux en comprenant que les autres personnes peuvent avoir des connaissances et sentiments différents des leurs et de prendre en compte ces états mentaux pendant la prise de décision. La théorie de l'esprit a beaucoup été étudiée dans les sciences sociales \cite{baron1985does, premack1978does}, notamment la notion de prise de perspective qui désigne la capacité d'une personne à prendre le point du vue d'une autre personne \cite{tversky1999speakers, flavell1992perspectives}. Deux niveaux de prise de perspective sont définis dans \cite{flavell1977development}. La prise de perspective perceptuelle désigne la capacité d'une personne à comprendre que les autres ont une représentation du monde différente de la sienne (fig~\ref{subfig:conceptual}). La prise de perspective conceptuelle désigne la capacité d'une personne à attribuer des croyances et connaissances à une autre personne (fig~\ref{subfig:conceptual}).

\begin{figure*}[!h]
    \centering
    \subfigure[Prise de perspective perceptuelle: deux individus peuvent avoir deux représentations différentes de leur environnement.]{
        \centering
        \includegraphics[width=0.4\textwidth]{figs/Chapter3/perceptual.jpg}
       \label{subfig:perceptual}
   }\hfill
    %~
    \subfigure[Prise de perspective conceptuelle: ici Bob attribue à Alice une connaissance concernant la boite: il pense qu'Alice pense que la boite est vide.]{
        \centering
        \includegraphics[width=0.4\textwidth]{figs/Chapter3/conceptual.jpg}
       \label{subfig:conceptual}
    }
    \caption{Illustration de la prise de perspective perceptuelle et conceptuelle.}
\end{figure*}

En robotique, plusieurs travaux ont pour but de doter le robot de capacités liées à la théorie de l'esprit. Un des premiers travaux sur ce sujet est celui de Scassellati où il propose un modèle pour adapter deux modèles des sciences sociales \cite{leslie1984spatiotemporal, baron1997mindblindness} afin d'implémenter la théorie de l'esprit en robotique \cite{scassellati2002theory}. Plusieurs travaux ont permis aux robots de se doter de capacités de prise de perspective \cite{berlin2006perspective, hiatt2010cognitive, milliez2014framework}. Ces capacités ont été utilisées dans plusieurs travaux visant par exemple à mieux reconnaître et comprendre les actions de l'Homme \cite{johnson2005perceptual, baker2014modeling, nagai2015probabilistic} ou pour résoudre des situations ambiguës \cite{breazeal2006using}. Des travaux ont été réalisés pour prendre en compte le point de vue de l'homme durant l'élaboration d'un plan partagé \cite{warnier2012robot}, cependant, aucun ne concerne l'exécution de ce plan. Cette partie de la thèse a pour but de commencer à combler ce manque.

\subsubsection{Estimation des états mentaux}

Dans un premier temps le robot doit être capable d'étendre l'estimation des états mentaux de ses partenaires (qui concernait précédemment les connaissances sur l'environnement) aux connaissances concernant la tâche en cours et le plan partagé. Les algorithmes développés permettent au robot d'estimer les états mentaux de l'Homme concernant :
\begin{itemize}
\item \textbf{l'état du monde :} en plus de l'estimation des connaissances de l'Homme concernant l'état du monde observable venant de la prise de perspective (e.g. un objet est sur un autre objet) le robot est capable d'estimer les connaissances de l'Homme concernant l'état du monde non-observable (e.g. une boite est vide ou remplie) en se basant principalement sur les effets des actions.
\item \textbf{le plan partagé :} en se basant sur ce que l'Homme peut observer, le robot est capable d'estimer ses connaissances concernant les actions en cours ou passées. Grâce à cette estimation et à ses propres connaissances concernant le plan partagé, le robot est capable d'estimer les connaissances de l'Homme concernant l'état du plan (e.g. quelles actions doivent être exécutées).
\item \textbf{le but :} le robot est capable d'estimer des connaissances basiques de l'Homme concernant l'état du but en cours (e.g. si il est achevé) en se basant sur ses connaissances sur l'état du monde.
\end{itemize}

\subsubsection{Utilisation des états mentaux durant l'exécution du plan partagé}

Une fois les états mentaux de ses partenaires estimés, le robot doit être capable de correctement les utiliser afin de communiquer durant l'exécution du plan partagé quand une différence apparaît entre les connaissances du robot et celles de l'Homme. En effet, le robot doit fournir à l'Homme les informations dont il a besoin pour réaliser la tâche sans pour autant être trop verbeux en informant l'Homme à propos de tout et n'importe quoi. Pour cela, nous avons développé plusieurs comportements pour le robot :
\begin{itemize}
\item en accord avec la notion de \textit{weak achievement goal} de \cite{cohen1991teamwork}, si le robot détecte une différence entre ses connaissances concernant l'état du but en cours et celle d'un de ses partenaires, le robot va informer ce partenaire à propos de cette différence.
\item  lorsque le robot estime qu'un de ses partenaires doit effectuer une action, il vérifie si il estime que ce partenaire sait qu'il doit réaliser l'action. Si ce n'est pas le cas, le robot cherche la raison de cette différence de croyance et communique à ce propos.
\item lorsque le robot estime qu'un de ses partenaires pense qu'il doit effectuer une action, il vérifie si il estime également que l'action doit être réalisée. Si ce n'est pas le cas, le robot cherche la raison de cette différence de croyance et communique à ce propos.
\item quand le robot s’apprête à réaliser une action, il vérifie qu'il estime que ses partenaires sont au courant de cette action, et si ce n'est pas le cas, le robot signale son action avant d'agir.
\item finalement, comme l'estimation des connaissances de l'Homme par le robot peut être erronée, si le robot estime que l'Homme a toutes les connaissances pour réaliser une action mais que l'Homme n'agit pas, le robot va simplement demander à l'Homme de réaliser l'action et considérer que son estimation était erronée.
\end{itemize}

\subsubsection{Exemple illustratif}


\begin{figure}[!h]
	\centering
    \includegraphics[width=0.6\textwidth]{figs/Chapter3/cleanWithNames.png}
    \caption{État du monde au début de la tâche de nettoyage de table. Le robot peut atteindre le \textit{grey book} et le \textit{white book} tandis que l'homme peut atteindre le \textit{white book} et le \textit{blue book}.}
    \label{fig:initClean}
\end{figure}


Pour illustrer les bénéfices de ce travail, nous avons utilisé une tâche ou un Homme et un robot doivent nettoyer une table ensemble. Pour cela, ils doivent enlever tous les objets initialement placés sur une table (Fig.~\ref{fig:initClean}), puis le robot doit balayer la table et enfin les objets doivent être remis sur la table. Le plan initialement produit par le robot pour atteindre ce but est celui Fig.~\ref{fig:initPlanClean}.

\begin{figure}[!h]
	\centering
    \includegraphics[width=0.9\textwidth]{figs/Chapter3/InitCleanPlan.png}
    \caption{Plan initial pour la tâche de nettoyage de table.}
    \label{fig:initPlanClean}
\end{figure}

Le robot commence à enlever le \textit{grey book} pour le placer sur le meuble à côté de lui. Pendant ce temps, l'Homme enlève le \textit{blue book} et s'en va (Fig.~\ref{subfig:humanLeave}). Le robot termine son action. A ce stade du plan, la prochaine action qui doit être effectuée est celle de l'Homme (enlever le \textit{white book}). Comme l'homme ne revient pas, le robot calcule un nouveau plan où il enlève le \textit{white book} (Fig.~\ref{fig:newplan}).

\begin{figure}[!h]
	\centering
    \includegraphics[width=0.9\textwidth]{figs/Chapter3/SecondCleanPlan.png}
    \caption{Deuxième plan pour la tâche de nettoyage de table.}
    \label{fig:newplan}
\end{figure}

Le robot enlève le dernier livre et balaye la table (Fig.~\ref{subfig:robotSweep}). L'Homme revient alors (Fig.~\ref{subfig:humanComesBack}). Comme il peut voir que le \textit{grey book} est sur le meuble à côté du robot, le robot estime que l'Homme est capable de déduire par lui même que le robot a fini sa première action (enlever le \textit{grey book}). De même, comme l'homme peut voir le \textit{white book}, le robot estime également que l'homme est au courant que le robot a enlevé ce livre. Cependant, l'Homme ne peut pas observer que la table a été balayée (on considère ici que la table n'était pas très sale et que l'effet de balayer la table n'est pas observable). Comme l'Homme a besoin de savoir que la table a été nettoyée pour remettre le livre qu'il avait enlevé, le robot va l'informer à propos de cette action ("J'ai balayé la table."). L'Homme a donc toutes les informations nécessaires pour finir la tâche (et aucune information superflue qu'il pouvait observer de lui même), lui et le robot finissent donc la tâche avec succès (Fig.~\ref{subfig:endClean}).


\begin{figure*}[!h]
    \centering
    \subfigure[L'Homme part après avoir enlevé le premier livre.]{
        \centering
        \includegraphics[width=0.4\textwidth]{figs/Chapter3/HumanLeave.png}
       \label{subfig:humanLeave}
   }\hfill
    %~
    \subfigure[Le robot enlève le dernier livre et balaye la table.]{
        \centering
        \includegraphics[width=0.44\textwidth]{figs/Chapter3/robotSweep.png}
       \label{subfig:robotSweep}
    }
    %~
    \subfigure[L'homme revient.]{
        \centering
        \includegraphics[width=0.4\textwidth]{figs/Chapter3/humanComesBack.png}
       \label{subfig:humanComesBack}
   }\hfill
    %~
    \subfigure[L'Homme et le robot finissent la tâche avec succès.]{
        \centering
        \includegraphics[width=0.45\textwidth]{figs/Chapter3/endClean.png}
       \label{subfig:endClean}
    }
    \caption{Exemple illustratif d'une tâche de nettoyage de table.}
\end{figure*}


\newpage
\subsection{Quand prendre les décisions pendant l'élaboration et l'exécution de plans partagés ?}


\subsubsection{Motivations}

Quand plusieurs individus collaborent lors d'une action conjointe, et plus particulièrement lors de l'exécution d'un plan partagé, de nombreuses décisions doivent être prises. Certaines d'entre elles vont être implicites alors que d'autres vont nécessiter une négociation ou une adaptation entre les acteurs de l'action conjointe. Afin d'être un bon partenaire, le robot doit donc être capable de prendre les bonnes décisions au bon moment et de correctement communiquer à leur propos (ne pas devenir trop verbeux en communiquant à propos des décisions implicites tout en donnant les informations nécessaires au bon déroulement de la tâche). Nous avons identifié trois types de décisions que le robot doit prendre durant l'élaboration et l'exécution d'un plan partagé :
\begin{itemize}
\item \textbf{Quelles actions exécuter dans quel ordre ?} cela a été le sujet de plusieurs travaux dans le domaine de l'interaction Homme-Robot. Nous utiliserons pour gérer ce type de décisions, HATP, un planificateur de tâche  capable de prendre en compte l'Homme \cite{Lallement2014hatp}.
\item \textbf{Qui doit effectuer chaque action ?} cette décision est quelques fois implicite quand une seule personne est capable d'exécuter une action, mais peut, dans certains cas, demander une négociation ou une adaptation de la part du robot. Dans les versions précédentes d'HATP, toutes ces décisions étaient prises à l'élaboration du plan. Un des objectifs de ce travail et de reporter cette décision à l'exécution quand elle n'est pas implicite afin de gagner en fluidité et en adaptabilité par rapport à l'Homme.
\item \textbf{Avec quels objets effectuer une action ?} il peut arriver que deux objets soit fonctionnellement équivalents dans le cadre de la tâche. Dans ce cas, le robot prenait auparavant à l'élaboration du plan une décision arbitraire quant à l'objet à utiliser durant une action. Le deuxième objectif de ce travail et de reporter cette décision à l'exécution quand il y a plusieurs objets fonctionnellement équivalents afin de gagner en fluidité et adaptabilité par rapport à l'Homme.
\end{itemize}


\subsubsection{Élaboration du plan partagé}

Afin de pouvoir reporter certaines décisions à l'exécution, nous avons effectué deux changements à la façon dont HATP élabore un plan :
\begin{itemize}
\item Afin de reporter la décision de qui doit effectuer une action quand plusieurs agents peuvent effectuer cette action, nous avons introduit dans HATP un agent virtuel appelé \textit{l'agent X}. Cette agent aura comme capacités l'intersection des capacités de l'Homme et du robot et aura un coût bien plus faible que les autres agents quand il réalisera une action. De cette manière, quand une action pourra être réalisée soit par l'Homme, soit par le robot, elle sera automatiquement attribuée à \textit{l'agent X} et la décision sera prise à l'exécution.
\item Nous avons également introduit ce que l'on a appelé des \textit{objets haut niveaux}. Ces \textit{objets haut niveaux} seront utilisés dans le plan par HATP quand deux objets fonctionnellement équivalents pourront être utilisés pour réaliser une action (par exemple CUBE$\_$ ROUGE à la place de CUBE$\_$ROUGE1 ou CUBE$\_$ROUGE2). 
\end{itemize}

\subsubsection{Exécution du plan partagé}

Une fois le plan élaboré, le robot doit être capable de l'exécuter en prenant les bonnes décisions au bon moment. Pour cela, grâce aux travaux antérieurs à cette thèse \cite{fiore2014planning}, le robot est capable de maintenir le plan partagé et de réagir aux actions inattendues de l'Homme. Quand le robot aura à effectuer une action du plan, il choisira en priorité une action qui lui est allouée par rapport à une action allouée à \textit{l'agent X} de manière à laisser le choix le plus longtemps possible à l'Homme d'effectuer cette action ou non. 

Quand le robot devra choisir qui doit réaliser une action allouée à \textit{l'agent X}, il vérifiera dans un premier temps quels sont les agents réellement disponibles pour effectuer cette action. Si le robot est le seul à pouvoir réaliser l'action (e.g. l'homme est déjà en train d'effectuer une autre action), il prendra l'initiative de réaliser l'action. Si l'Homme et le robot peuvent tous les deux réaliser l'action, le robot aura alors deux différents modes possibles :
\begin{itemize}
\item \textbf{Le mode négociation :} où le robot demande à l'homme si il veut réaliser l'action et agit (ou non) en fonction de sa réponse.
\item \textbf{Le mode adaptation :} où le robot attend un temps court de voir si l'homme prend l'initiative de réaliser l'action, et, si ce n'est pas le cas, prend lui même l'initiative de la réaliser.
\end{itemize}
Une fois une action allouée par le robot, il calcule un nouveau plan pour vérifier que cette allocation n'a pas d'autres implications dans le plan.

Finalement, quand le robot doit réaliser une action comportant un \textit{objet haut niveau}, le robot va reporter la décision au dernier moment possible pour laisser plus de latitude à l'Homme. Pour prendre la décision de quel objet utiliser, le robot pourra utiliser des coûts prenant en compte l'Homme tels que la distance entre les agents et les différents objets. L'exécution de l'action par le robot se fera en boucle fermée et en surveillant l'activité de l'Homme de manière à pouvoir changer de décision si l'Homme prend une initiative en conflit avec la décision précédente.

\subsubsection{Exemple illustratif}


\begin{figure*}[!h]
\centering
	\subfigure[But de la tâche (vue de côté).]{
        \centering
        \includegraphics[width=0.25\textwidth]{figs/Chapter4/BlockGoal.png}
       \label{subfig:goal}
   }\hfill
    %~
	\subfigure[Un possible état de départ (vue de haut).)]{
        \centering
        \includegraphics[width=0.4\textwidth]{figs/Chapter4/SetUp.png}
       \label{subfig:setUp}
   }
    \caption{Description de la tâche de construction de blocs. L'Homme et le robot doivent construire une pile ensemble.}
    \label{fig:blocksBuildingTask}
\end{figure*}

Pour illustrer les bénéfices de ce travail, nous avons utilisé une tâche inspirée de celle présentée dans \cite{clodic2014key}. Un Homme et un robot doivent réaliser une construction avec des blocs colorés comme représenté Fig.~\ref{subfig:goal}. Au début de la tâche, l'Homme et le robot ont chacun plusieurs blocs de couleur à leur disposition comme par exemple Fig. \ref{subfig:setUp}. Deux emplacements identiques sont placés au centre de la table pour indiquer ou mettre les deux premiers cubes rouges.

\begin{figure*}[!h]
\centering
	\subfigure[Plan initial.]{
        \centering
        \includegraphics[width=0.7\textwidth]{figs/Chapter4/second_plan.png}
       \label{subfig:initPlan}
   }
    %~
	\subfigure[Le robot choisit de mettre son cube rouge sur l'emplacement à sa droite.]{
        \centering
        \includegraphics[width=0.4\textwidth]{figs/Chapter4/screen_shot1.jpeg}
       \label{subfig:redCube}
   }\hfill
    %~
	\subfigure[L'Homme pose son cube sur l'emplacement que le robot a choisit.]{
        \centering
        \includegraphics[width=0.4\textwidth]{figs/Chapter4/screen_shot2.jpeg}
       \label{subfig:humanPlace}
   }
    %~
	\subfigure[Le robot s'adapte en changeant son choix d'emplacement.]{
        \centering
        \includegraphics[width=0.4\textwidth]{figs/Chapter4/screen_shot3.jpeg}
       \label{subfig:robotAdapts}
   }\hfill
    %~
	\subfigure[Deuxième plan.]{
        \centering
        \includegraphics[width=0.4\textwidth]{figs/Chapter4/third_plan.png}
       \label{subfig:thirdPlan}
   }
    \caption{Exemple illustratif d'une tâche de construction de blocs.}
\end{figure*}

Le plan produit initialement par HATP pour réaliser la tâche peut être trouvé Fig.~\ref{subfig:initPlan}. Le robot attrape le cube rouge à sa disposition et choisit de le placer sur l'emplacement à sa droite (Fig. \ref{subfig:redCube}). Cependant, l'Homme prend son cube rouge et choisit de le placer sur le même emplacement que celui choisi par le robot (Fig.~\ref{subfig:humanPlace}). Le robot interrompt son action et s'adapte en plaçant son cube sur l'autre emplacement (Fig.~\ref{subfig:robotAdapts}). L'homme pose alors le bâton jaune sur les cubes rouges. Dans cet exemple, le robot utilise le mode \textbf{négociation} pour choisir qui va mettre le premier cube bleu. Le robot demande donc à l'Homme si il veut poser le cube bleu ("Voulez-vous poser le cube bleu ?"). L'Homme répond oui, le robot calcule donc un nouveau plan ou il posera le second cube bleu Fig.~\ref{subfig:thirdPlan}. Finalement, l'Homme et le robot effectuent leurs dernières actions et réalisent la tâche avec succès.

\newpage
\subsection{Évaluation du système}

\subsubsection{Tâche et conditions}

Afin d'évaluer le nouveau système développé concernant la gestion des plans partagés par le robot, nous avons développé une tâche d'inventaire. Dans cette tâche, l'Homme et le robot doivent scanner différents cubes de couleur et les ranger dans une boîte de même couleur. Au début de la tâche chaque agent a une pile de cubes de différentes couleurs à laquelle seulement lui peut accéder. Ces piles contiennent des cubes bleus, verts et rouges. Pour que les cubes soit scannés, les agents doivent les poser un par un sur une des deux zones de scan sur la table devant le robot (voir Fig.~\ref{fig:setUpSimu}). Une fois un cube sur une zone, le robot peut le scanner en orientant sa tête vers le cube et en allumant une lumière rouge en direction du cube (voir Fig.~\ref{fig:scan}). Une fois un cube scanné, il peut être rangé dans une boite de la même couleur. Le robot a accès à une boite bleue et à une boite rouge tandis que l'Homme a accès à une boite verte et à une boite rouge (voir Fig.~\ref{fig:setUpSimu}). Cette tâche comporte deux particularités intéressantes pour notre système:
\begin{itemize}
\item Comme la pile de l'homme et ses boîtes sont situées dans des pièces différentes que celle du robot (voir Fig.~\ref{fig:setUpSimu}), si le robot scanne un cube quand l'Homme est parti chercher ou ranger un cube, l'Homme ne pourra pas savoir que le cube a été scanné sauf si le robot l'en informe (pas d'effets visibles).
\item La répartition des boites entre les agents fait que seul le robot peut ranger les cubes bleus, seul l'Homme peut ranger les cubes verts mais qu'ils peuvent tous les deux ranger les cubes rouges.
\end{itemize}

\begin{figure}[!h]
	\centering
    \includegraphics[width=0.9\textwidth]{figs/Chapter5/SetUpSimu.png}
    \caption{Etat initial pour la tâche d'inventaire. L'Homme et le robot doivent prendre les cubes de leur pile pour les mettre sur une zone de scan. Puis, le robot doit scanner le cube et enfin chaque cube doit être rangé dans une boite de même couleur. L'Homme a accès à une boite verte et à une boite rouge tandis que le robot a accès à une boite bleue et à une boite rouge.}
    \label{fig:setUpSimu}
\end{figure}

Nous avons évalué notre système en simulation et lors d'une étude utilisateur. Pour faire cela, nous avons comparé 4 différentes conditions :
\begin{itemize}
\item avec le système original, appelé \textbf{système de référence (RS)}, où toutes les décisions du robot sont prises durant l'élaboration du plan et sans estimation des états mentaux de l'Homme:
\begin{itemize}
\item \textbf{RS-NONE mode :} le robot ne verbalise rien (sauf en cas de stricte nécessité).
\item \textbf{RS-ALL mode :} le robot informe à propos de toutes les actions qu'il doit faire et que l'homme doit réaliser ainsi qu'à propos de toutes les actions manquées par l'Homme.
\end{itemize}
\item avec le nouveau système développé présenté précédemment (\textbf{NS}) :
\begin{itemize}
\item \textbf{NS-N mode :} le robot utilise le mode \textbf{négociation} pour prendre une décision concernant les actions de \textit{l'agent X}.
\item \textbf{NS-A mode :} le robot utilise le mode \textbf{adaptation} pour prendre une décision concernant les actions de \textit{l'agent X}.
\end{itemize}
\end{itemize}

\subsubsection{Évaluation en simulation}

Pour évaluer notre système en simulation, nous avons fait tourner la tâche avec différents états de départ ou les piles des agents étaient aléatoirement composées. Durant ces simulations, le robot était confronté à plusieurs types d'Hommes simulés :
\begin{itemize}
\item \textit{l'Homme "aimable" (cas \textbf{K}) :} qui adapte son comportement à ce que verbalise le robot. Concernant les cubes rouges, il peut choisir de ne jamais les ranger (\textbf{lazy-K}), les ranger systématiquement (\textbf{hurry-K}) ou les ranger avec une probabilité de 50\% (\textbf{50\%-K})
\item \textit{l'Homme "têtu" (cas \textbf{S}) :} qui n'adapte pas son comportement à ce que verbalise le robot. Concernant les cubes rouges, il peut choisir de ne jamais les ranger (\textbf{lazy-S}), les ranger systématiquement (\textbf{hurry-S}) ou les ranger avec une probabilité de 50\% (\textbf{50\%-S})
\end{itemize}
Dans tous les cas, l'Homme participe activement au dépôt des cubes de sa pile sur les zones de scan et au rangement des cubes verts et répond aux questions posées par le robot.

Les données mesurées durant ces simulations sont :
\begin{itemize}
\item \textit{le nombre d'interactions verbales :} entre l'Homme et le robot (information donnée par le robot ou question posée), Tab.~\ref{tab:verbalInteraction}.
\item \textit{le nombre de décisions incompatibles :} les deux acteurs prennent la même décision concernant une action (les deux essayent de la réaliser ou de ne pas la réaliser), Tab.~\ref{tab:incompatibleDecisions}.
\item \textit{le temps d'exécution total :} pour réaliser la tâche, Fig.~\ref{fig:resTime}.
\end{itemize}

\begin{table*}[!h]
\centering
  \begin{tabular}{|c||c|c|c|c|}
  \hline
     & \textbf{RS-NONE} & \textbf{RS-ALL} & \textbf{NS-N} & \textbf{NS-A} \\
  \hline
  \hline
     \textbf{50\%-K} & 2.4 (0.84) & 20.7 (1.34) & 3.4 (1.51) & 2 (1.33) \\
  \hline
     \textbf{hurry-K} & 1.8 (0.79) & 21.1 (2.08) & 1.9 (1.10) & 2.2 (1.13) \\
  \hline
     \textbf{lazy-K} & 3.0 (1.33) & 21 (1.56) & 3.3 (1.42) & 1.6 (1.17) \\
  \hline
     \textbf{50\%-S} & 2.5 (1.43) & 23.9 (1.59) & 3.3 (1.49) & 1.7 (0.95)\\
  \hline
     \textbf{hurry-S} & 1.5 (0.97) & 20.9 (1.29) & 2.4 (1.89) & 1.9 (0.99)\\
  \hline
     \textbf{lazy-S} & 3.2 (0.92) & 25.2 (1.55) & 2.8 (1.68) & 1.8 (1.14)\\
  \hline
  \end{tabular}
   \caption{\textbf{Nombre d'interactions verbales} : questions posées par le robot dans le mode négociation et nombre d'informations verbalisées. Ces résultats correspondent à la moyenne sur 10 essais et leur déviation standard associée.}
   \label{tab:verbalInteraction}
\end{table*}

Plusieurs choses peuvent être observées par rapport aux résultats obtenus :
\begin{itemize}
\item le mode \textbf{RS-NONE} est celui comportant le plus de décisions incompatibles (du fait que le robot ne communique et de s'adapte pas par rapport aux cubes rouges et aux choix des zones de scan). Ce mode comporte également les plus grands temps d'exécution, plus spécialement dans le cas de l'Homme "têtu" car le robot perd du temps à attendre qu'il exécute des actions qu'il ne veut pas réaliser ou à attendre que l'Homme range un cube dont il ne sait pas qu'il a été scanné.
\item comme attendu, le mode \textbf{RS-ALL} est celui avec le plus d'interactions verbales. Cependant, ces interactions verbales ne suffisent pas à supprimer toutes les décisions incompatibles, surtout dans le cas de l'homme "têtu" ou le temps d'exécution est également plus élevé.
\item on peut voir que les performances du \textbf{nouveau système} sont globalement meilleures que celles de l'ancien système. En effet pour beaucoup moins d'informations verbalisées, il permet de supprimer les décisions incompatibles et de réduire le temps d'exécution dans le cas de l'Homme "têtu". Le mode \textbf{adaptation} obtient les mêmes résultats que le mode \textbf{négociation} mais avec moins d'interactions verbales.
\end{itemize}


\begin{table*}[!h]
\centering
  \begin{tabular}{|c||c|c|c|c|}
  \hline
     & \textbf{RS-NONE} & \textbf{RS-ALL} & \textbf{NS-N} & \textbf{NS-A} \\
  \hline
  \hline
     \textbf{50\%-K} & 2.9 (0.99) & 0.9 (0.57) & 0.6 (0.7) & 0.3 (0.48) \\
  \hline
     \textbf{hurry-K} & 2.5 (0.97) & 1.0 (0.94) & 0.6 (0.52) & 0.4 (0.52)\\
  \hline
     \textbf{lazy-K} & 3.5 (1.08) & 0.8 (0.63) & 0.5 (0.7) & 0.5 (0.53) \\
  \hline
     \textbf{50\%-S} & 2.9 (1.45) & 1.9 (0.99) & 0.6 (0.52) & 0.5 (0.97) \\
  \hline
     \textbf{hurry-S} & 2.3 (1.34) & 1.0 (0.82) & 0.5 (0.53) & 0.4 (0.52) \\
  \hline
     \textbf{lazy-S} & 3.5 (0.97) & 2.6 (1.84) & 0.3 (0.67) & 0.4 (0.52)  \\
  \hline
  \end{tabular}
   \caption{\textbf{Nombre de décisions incompatibles} : les deux acteurs prennent la même décision concernant une action (les deux essayent de la réaliser ou de ne pas la réaliser). Ces résultats correspondent à la moyenne sur 10 essais et leur déviation standard associée.}
   \label{tab:incompatibleDecisions} 
\end{table*}

\begin{figure}[!h]
	\centering
    \includegraphics[width=0.9\textwidth]{figs/Chapter5/Time.png}    \caption{Temps en secondes nécessité pour chaque système pour réaliser la tâche dans chaque condition (moyennes sur 10 essais).}
    \label{fig:resTime}
\end{figure}

\subsubsection{Étude utilisateur}

\begin{figure}[!t]
	\centering
    \includegraphics[width=0.7\textwidth]{figs/Chapter5/Scan.png}
    \caption{Le robot PR2 interagissant avec un sujet pour réaliser la tâche. Le robot scanne un objet avant de le ranger.}
    \label{fig:scan}
\end{figure}


\paragraph{Adaptation de la tâche pour l'étude utilisateur:}

Avant de réaliser l'étude utilisateur, nous avons effectué quelques pré-tests qui nous ont permis d'identifier plusieurs problèmes et d'y remédier avec des adaptations de la tâche :
\begin{itemize}
\item \textbf{introduction d'une cassette rouge :} afin d'assurer qu'il y ait forcement une prise de décision par rapport à un objet rouge (de temps en temps la configuration faisait qu'aucune décision n'était nécessaire), nous avons ajouté une cassette rouge qui doit être scannée et rangée une fois que tous les cubes ont été rangés. L'Homme et le robot ont chacun initialement une cassette rouge mais une seule doit être scannée et rangée.
\item \textbf{Tâche de distraction :} afin d'être surs d'obtenir un manque de connaissance à un moment de la tâche (de temps en temps l'homme ne s'absentait jamais lors d'une action de scan), nous avons rajouté une tâche de construction avec des Légos pour le sujet à un moment de la tâche dans un lieu ou il ne peut pas voir le robot.
\end{itemize}


\paragraph{Questionnaire et protocole:}

21 sujets (8 femmes et 13 hommes) ont interagi avec le robot pour réaliser la tâche dans les quatre conditions décrites précédemment. L'ordre de ces conditions et les compositions des piles des agents étaient aléatoires.
A leur arrivée, les participants étaient introduits à l'environnement de travail et au robot par l’expérimentateur. Ensuite, les participants avaient à lire les consignes de la tâche et l’expérimentateur vérifiait leur bonne compréhension. Les participants réalisaient une rapide tâche de familiarisation avant de réaliser la vraie tâche.
Après chaque interaction avec le robot (pour chaque condition), les participants avaient à remplir un questionnaire leur permettant d'évaluer le comportement du robot. Comme nous n'avons pas trouvé dans la littérature existante de questionnaires permettant d'évaluer la prise de décision haut niveau d'un robot lors d'une tâche de collaboration avec l'Homme, nous avons conçu ce questionnaire en nous basant sur le modèle d'expérience utilisateur UX \cite{mahlke2008user} et en ajoutant des dimensions spécifiques à la prise de décision. Ce questionnaire est composé de plusieurs dimensions :
\begin{itemize}
\item \textbf{Dimension de collaboration}, basée sur \cite{weistroffer2014etude} et permettant d'évaluer la perception de l'utilité et de l'utilisabilité du robot \cite{davis1989perceived}.
\item \textbf{Dimension d'interaction}, basée sur \cite{lallemand2015creation} et permettant d'évaluer l'intention d'utilisation \cite{davis1989perceived}.
\item \textbf{Dimension de perception du robot}, basée sur le questionnaire Godspeed \cite{bartneck2009measurement} et permettant d'évaluer comment le sujet perçoit le robot en général \cite{hassenzahl2003thing}.
\item \textbf{Dimension émotions,} reprise de l'AffectButton \cite{broekens2013affectbutton} et permettant d'évaluer les émotions du sujet lors de l'interaction.
\item \textbf{Dimension verbale} permettant d'évaluer comment le sujet a perçu les interactions verbales avec le robot.
\item \textbf{Dimension d'action} permettant d'évaluer comment le sujet a perçu la prise de décision du robot par rapport au choix d’exécution des actions.
\end{itemize}
Ces dimensions étaient évaluées grâce à des questions où le sujet devait se placer sur une échelle de 100, sauf pour la dimension émotion ou le sujet avait à choisir entre plusieurs smileys.

\paragraph{Hypothèses:} 
Nous avons émis plusieurs hypothèses avant l'étude :
\begin{itemize}
\item \textbf{Hypothèse 1 :} le nouveau système sera préféré par les utilisateurs à l'ancien système.
\item \textbf{Hypothèse 2 :} concernant le nouveau système, au contraire des résultats de simulation, le mode négociation sera préféré par les utilisateurs au mode adaptation.
\end{itemize}

\paragraph{Résultats:}

La cohérence interne de questionnaire a été vérifiée à la suite de cette étude (alpha de Cronbach supérieur à 0,7 pour toutes les dimensions du questionnaire). Concernant les scores des différentes conditions, les résultats totaux du questionnaire peuvent être trouvés en Fig~\ref{fig:resUSTotal}.

\begin{figure}[!h]
	\centering
    \includegraphics[width=0.7\textwidth]{figs/Chapter5/Total.png}
    \caption{Scores totaux du questionnaire utilisateur. Addition des scores de toutes les dimensions précédemment remis sur une échelle de 100.}
    \label{fig:resUSTotal}
\end{figure}

Les scores du nouveau système (NS) ont été trouvés significativement plus élevés (p < 0.05) que ceux de l'ancien système (RS). Cela permet donc de vérifier la première hypothèse comme quoi le nouveau système a été préféré des sujets. Concernant le nouveau système, si l'on regarde la dimension verbale du questionnaire, le score de la condition \textbf{négociation} a été trouvé significativement plus élevé que celui de la condition \textbf{adaptation} (p < 0.05). Il n'y a pas eu de différence significative concernant les autres dimensions. Comme la seule différence entre ces deux modes consiste à poser ou non une question quand il y avait une décision à prendre concernant un objet rouge (comportement verbal), nous pouvons donc considérer la seconde hypothèse comme validée également.

\subsubsection{Conclusion}

L'évaluation du système en simulation et lors d'une étude utilisateur a permis de montrer que le nouveau système développé a de meilleures performances et une meilleure appréciation par l'utilisateur que l'ancien système. Concernant les deux modes possible du nouveau système, des utilisateurs naïfs comme ceux de l'étude utilisateur préfère le mode \textbf{négociation}. Cependant, pour des utilisateurs plus experts, le mode \textbf{adaptation} a montré de meilleurs résultats en simulation. L'étude utilisateur nous a également permis de développer et valider un questionnaire permettant d'évaluer la prise de décision haut niveau du robot lors d'une tâche de collaboration avec l'Homme. 


\newpage
\section{Autres contributions à l'action conjointe Homme-Robot}

\subsection{Communication non-verbale: qu'est ce que le robot doit faire avec sa tête ?}

\subsubsection{Motivations et précédents travaux}

Pour communiquer entre eux quant ils collaborent sans être trop verbeux, les hommes utilisent fréquemment la communication non-verbale \cite{ekman1969repertoire, depaulo1992nonverbal}. Durant l'action conjointe Homme-Robot, le robot doit également être capable de communiquer à son partenaire toutes les informations dont il a besoin sans être trop intrusif. Pour cela, il doit donc être capable d'avoir un comportement non-verbal adapté à l'action conjointe. La communication non-verbale vient de multiples sources (expression faciales \cite{labarre1947cultural}, postures \cite{mehrabian1969significance}, regard \cite{mutlu2009designing}, etc...). Dans cette thèse, nous nous sommes concentré sur l'utilisation de la tête du robot, remplaçant les signaux donnés par le regard en l'absence de pupilles pour le robot \cite{imai2002robot}.

Les sciences sociales ont permis de déterminer plusieurs utilisations du regard durant l'action conjointe entre Hommes :
\begin{itemize}
\item \textbf{Aide au dialogue et à la prise de tour :} le regard est très utilisé lors du dialogue \cite{argyle1976gaze} et plus spécifiquement pour signaler les changements d'orateur \cite{kendon1967some}. 
\item \textbf{Aide à la compréhension des actions :} les acteurs d'une action conjointe vont agir différemment que lorsqu'ils agissent seul \cite{becchio2010toward, vesper2010minimal}. Plus spécifiquement, l'utilisation du regard lors d'une action va permettre aux partenaires de mieux interpréter les intentions de l'acteur \cite{castiello2003understanding, pierno2006gaze}.
\item \textbf{Aide à la compréhension des états mentaux :} l'observation du regard du partenaire permet de mieux prendre sa perspective afin de mieux estimer ses connaissances \cite{furlanetto2013through}.
\end{itemize}

En robotique, plusieurs études ont montré l’intérêt du comportement non-verbal du robot \cite{furlanetto2013through, haring2012studies}. Beaucoup de travaux se sont concentrés sur l'utilisation de la tête lors du dialogue \cite{mutlu2009footing, boucher2010facilitative, skantze2014turn}. Seuls quelques uns portent sur l'utilisation du regard durant l'action conjointe et ont montré qu'un bon comportement de la part du robot aide à la coordination lors de l'exécution du plan partagé \cite{lallee2013cooperative} et à la prise de décision de l'Homme \cite{boucher2012reach}.

\subsubsection{Réflexion concernant les signaux et comportements nécessaires}

Sur la base d'une étude bibliographique des comportements humains et des travaux sur la tête du robot, nous avons identifié ce que nous pensons être des composants nécessaires à un bon comportement de la tête du robot lors de l'action conjointe :
\begin{itemize}
\item \textbf{Lorsque le robot agit :} lors de ses actions, le robot doit utiliser sa tête à la fois pour la bonne réalisation de l'action d'un point de vue fonctionnel (présence de caméras à l'intérieur de la tête) mais également pour indiquer à ses partenaires ce qu'il fait et ce qu'il va faire ensuite. 
\item \textbf{Lorsque le robot parle :} lors d'un dialogue, il est important pour le robot de regarder l'homme au bon moment ainsi que les objets dont il parle.
\item \textbf{Le robot observe :} le robot doit se servir de sa tête pour montrer son intérêt et sa compréhension des actions de l'Homme.
\item \textbf{Le robot se coordonne :} le robot doit fournir les signaux appropriés et nécessaires au bon déroulement du plan partagé.
\end{itemize}

\subsubsection{Étude approfondie de certains signaux }

Nous avons étudié dans plus de détails certains composants des comportements définis précédemment. Pour faire cela, nous avons mené une étude utilisateur en ligne à base de vidéos. Dans cette étude, nous avons demandé à 59 personnes (30 femmes et 29 hommes) de regarder plusieurs vidéos courtes ou le comportement de la tête du robot changeait et d'évaluer ces comportements grâce à un petit questionnaire. Dans ces vidéos, l'Homme et le robot avaient à construire une pile de cubes colorés (comme illustré Fig.~\ref{fig:videoTask}).

\begin{figure*}[!t]
\centering
	\subfigure[Début de la tâche. Le cube bleu et le cube vert sont accessibles par l'Homme et les cubes noir et rouge sont accessibles par le robot.]{
        \centering
        \includegraphics[width=0.45\textwidth]{figs/Chapter6/VideoInit.png}
       \label{subfig:videoInit}
   }\hfill
    %~
	\subfigure[Fin de la tâche. La pile doit être construite dans un ordre précis (rouge, noir, bleu, vert).]{
        \centering
        \includegraphics[width=0.45\textwidth]{figs/Chapter6/VideoGoal.png}
       \label{subfig:videoGoal}
   }
    \caption{Tâche utilisée dans l'étude utilisateur en ligne. Dans cette tâche, l'Homme et le robot doivent construire une pile de cubes colorés.}
    \label{fig:videoTask}
\end{figure*}

Les comportements testés et les résultats obtenus sont les suivants :
\begin{itemize}
\item \textbf{Anticipation des actions du robot :} nous avons comparé un comportement du robot ou il anticipait sa prochaine action avec sa tête (il regarde le cube à prendre avant de commencer son action) au même comportement sans cette anticipation. Nous n'avons pas trouvé de différence significative entre ces deux comportements. En effet, certains des sujets étaient perturbés par le fait que le robot regarde le cube avant d'agir et d'autres n'ont pas vu la différence. Une possible explication pour cela est que la tâche ne demandait pas d'anticipation de la part du robot car les deux participants savaient quelle action était nécessaire à chaque moment (ordre de la pile prédéfini).
\item \textbf{Suivre l'activité de l'homme :} nous avons comparé différents moyens pour le robot de suivre avec sa tête l'activité de l'Homme. Dans la première condition, le robot regardait la main de l'homme dès qu'elle était en mouvement et la tête sinon. Dans la seconde, le robot regardait la main de l'Homme quand elle était dans une zone de travail définie au dessus de la table et la tête sinon. Finalement, dans la dernière condition, le robot regardait la main de l'Homme quand elle était en mouvement et dans la zone de travail et la tête sinon. Cette dernière condition a été significativement préférée aux deux autres par les sujets.
\item \textbf{Montrer la compréhension des actions de l'Homme :} nous avons comparé un comportement ou le robot marquait un arrêt avec sa tête quand l'Homme réalisait une action de manière à montrer qu'il avait détecté l'action à une condition sans cet arrêt. Nous n'avons pas trouvé de différence significative entre ces deux conditions, les sujets ayant du mal à trouver les différences entre les deux vidéos.
\item \textbf{Gérer l'inaction de l'Homme :} dans ces vidéos, l'Homme mettait du temps à réaliser une de ses actions. Nous avons comparé trois différentes réaction de la part du robot. Dans la première, le robot ne changeait pas son comportement de base face à cette inaction. Dans la seconde, le robot donnait un signal à l'Homme avec sa tête en regardant le cube que l'Homme devait prendre. Dans la dernière, le robot donnait un signal similaire mais cette fois ci regardait le cube de l'Homme puis la pile. Les deux conditions ou le robot donnait un signal à l'Homme ont été notées significativement mieux que celle sans signal montrant l'importance du signal du robot. Aucune différence n'a été trouvée entre les deux différents signaux.
\item \textbf{Aide à la prise de tour :} nous avons comparé différentes manières pour le robot de gérer le changement d'acteur dans la tâche (passage d'une action du robot à une action de l'Homme). Dans deux conditions, le robot ne donnait pas de signal particulier à l'Homme. Il regardait simplement l'Homme soit à la fin de son action dans une condition, soit après s'être retiré de son action dans l'autre condition. Dans les deux autres conditions, le robot regardait le cube que l'Homme devait poser avant de regarder l'Homme. Comme pour les deux précédentes conditions, le robot faisait cela à la fin de son action dans une condition et après s'être retiré dans l'autre condition. Les deux conditions où le robot regardait le cube de l'Homme ont été trouvées significativement meilleures que les deux autres, montrant l’intérêt du signal du robot. Aucune différence n'a été trouvée concernant le timing du signal (avant ou après le retrait).
\item \textbf{Choisir un objet d'attention :} dans le dernier scénario, l'Homme commençait à prendre un cube pendant que le robot était toujours en train de poser le sien. Dans une condition, le robot continuait son action sans regarder l'Homme. Dans la seconde condition, le robot regardait l'Homme mais sans interrompre sa propre action. Dans la dernière condition, le robot interrompait son action pour regarder celle de l'Homme. La condition ou le robot ne regarde pas l'Homme a été trouvée significativement moins bonne que les deux autres. Aucune différence n'a été trouvée entre les deux autres conditions. Cela montre l'importance pour le robot de regarder l'action de l'Homme même si il doit pour cela interrompre sa propre action.
\end{itemize}

\subsubsection{Proposition d'architecture pour le comportement de la tête du robot}

A partir de l'étude bibliographique et des résultats de l'étude présentée précédemment, nous avons proposé une architecture pour gérer la tête du robot. Cette architecture peut être trouvée Fig.~\ref{fig:headArchi}.

\begin{figure}[!h]
	\centering
    \includegraphics[width=0.9\textwidth]{figs/Chapter6/Head_archi.png}
    \caption{Architecture pour gérer la tête du robot. Un module d'arbitrage choisit où le robot doit regarder en se basant sur plusieurs comportements et signaux produits par des modules en amont.}
    \label{fig:headArchi}
\end{figure}

Un premier module permet au robot de générer un comportement pour montrer sa compréhension de l'activité de l'Homme. Basé sur les résultats de l'étude précédente, nous proposons d'implémenter un comportement où le robot regarde successivement la tête et la main de l'Homme en se basant sur le mouvement de la main et sa position (dans ou en dehors de zones de travail). Nous proposons également d'implémenter un comportement qui permet au robot de répondre aux regards de l'Homme (le robot regarde l'Homme si il le regarde et regarde l'objet de l'attention de l'Homme si l'Homme regarde fixement un objet).

En entrée de l'architecture proposée, nous trouvons des points d’intérêt venant du module d’exécution d'action du robot et du module de dialogue. Ces deux modules fournissent tous les deux l'objet ou la personne la plus pertinente à regarder en fonction de l'action du robot et de la conversation en cours.

Un autre module permet de créer des signaux à donner à l'Homme concernant l'exécution du plan partagé. Nous proposons d'implémenter les deux signaux étudiés précédemment (signal quand l'Homme n'agit pas et signal d'aide à la prise de tour). 

Finalement, un module d'arbitrage permet de choisir entre les différents comportements et signaux générés par les autres modules en se basant sur ce que fait le robot et les différentes priorités des signaux et comportements.

\newpage
\subsection{Combiner apprentissage et planification}

\subsubsection{Motivations et travaux précédents}

Concernant la prise de décision en robotique, on retrouve deux grandes écoles de pensée qui ont chacune leurs avantages et désavantages : l'apprentissage et les processus déterministes (ou planification). L'apprentissage est généralement "peu coûteux" au sens où une décision est prise rapidement et une solution va toujours être proposée quelque soit le problème. Cependant, la phase d'apprentissage requière une grande quantité de données et/ou une longue période d'apprentissage durant laquelle le robot va produire des comportements inconsistants et perturbants pour l'utilisateur humain. La planification va être plus lente à prendre une décision, particulièrement dans le cas d'un environnement ou d'une tâche complexe, mais va pouvoir prendre en compte des règles sociales et assurer la validité de la solution proposée dans son ensemble. L'idée de ce travail est de combiner ces deux techniques dans le contexte de le prise de décision pour l’interaction Homme-Robot.

Ces deux écoles de pensées sont inspirées des différents comportements des mammifères et de l'Homme : le comportement \textit{dirigé vers un but} pour la planification et le comportement \textit{habituel} pour l'apprentissage \cite{dickinson1985actions}. Différentes études ont été menées en neuroscience pour trouver comment alterner entre ces comportements \cite{pezzulo2013mixed, lesaint2014modelling, viejo2015modeling}. En robotique, de nombreux travaux ont été réalisés en planification \cite{ingrand2014deliberation} et en apprentissage \cite{kober2011learning, martins2010learning, stulp2013robot} mais peu d'entre eux se concentrent sur comment combiner ces approches.


\subsubsection{Présentations des différents experts}

\begin{figure}[!h]
	\centering
    \includegraphics[width=0.9\textwidth]{figs/Chapter7/SharedPlan.png}
    \caption{Un exemple de plan produit par HATP. Ce plan permet à un Homme et à un robot de nettoyer une table en enlevant tous les objets dessus, la nettoyant puis replaçant tous les objets enlevés précédemment.}
    \label{fig:examplePlan}
\end{figure}

Dans le travail présenté dans cette thèse, deux experts ont été utilisés pour modéliser les deux comportements évoqués précédemment :
\begin{itemize}
\item Le comportement \textit{dirigé vers un but} est fourni par HATP \cite{Lallement2014hatp}, un planificateur HTN conçu pour le contexte de l'interaction Homme-robot. HATP prend en compte les préconditions et effets des différentes actions possibles pour construire un plan qui permet d'atteindre un but précis depuis un contexte donné (e.g. Fig.~\ref{fig:examplePlan}). HATP permet de calculer un plan complet qui permet d'atteindre un but donné et qui prend en compte des coûts concernant l'Homme. Cependant, il ne permettra pas d'apprendre du comportement de l'Homme en direct, les coûts étant codés à l'avance. Son temps de décision sera plus lent que celui de l'autre expert mais ne nécessite pas de période d'apprentissage.
\item Le comportement \textit{habituel} est produit par un algorithme d'apprentissage par renforcement sans modèle \cite{renaudo2014design} qui permet d'apprendre une action à exécuter pour chaque état possible en se basant sur un principe de récompense. Cet algorithme est implémenté comme un réseau neuronal (voir Fig.~\ref{fig:Qlearning}). Cet algorithme permet de toujours proposer rapidement une action à exécuter par le robot. Cependant une longue phase d'apprentissage est nécessaire pendant laquelle le robot aura un comportement inconsistant au début et à chaque changement de la tâche.
\end{itemize}

\begin{figure}[!h]
	\centering
    \includegraphics[width=0.6\textwidth]{figs/Chapter7/Qlearning.png}
    \caption{L'expert du comportement habituel est un algorithme d'apprentissage implémenté comme un réseau neuronal. Il reçoit en entrée un état $S$ qui est projeté sur un neurone d'entrée $s_i$ définissant une activité d'entrée. L'activité est propagée grâce aux poids du réseau neuronal $W$ pour générer une activité sur le niveau d'action. Cette activité correspond à la valeur $Q(S, a_j)$ et est convertie en une probabilité de distribution permettant à l'expert de prendre une décision $D$ sur la prochaine action à exécuter.}
    \label{fig:Qlearning}
\end{figure}



\subsubsection{Première architecture : une preuve de concept}


\paragraph{Architecture :}

La première architecture développée pour combiner les deux experts peut être trouvée Fig.~\ref{fig:FirstArchi}. Dans cette architecture les deux experts sont placés en parallèle. Le module d'évaluation de la situation prend les données de la perception et maintient l'état du monde courant. Cet état du monde est utilisé par la supervision pour calculer la récompense et par les experts pour prendre une décision. Les propositions des deux experts sont envoyées au méta-contrôleur qui décide de l'action à exécuter (de manière aléatoire). Le superviseur exécute l'action avec l'aide des modules de plus bas niveau.

\begin{figure}[!h]
	\centering
    \includegraphics[width=0.8\textwidth]{figs/Chapter7/FirstArchi.png}
    \caption{Première architecture développée pour combiner les deux experts.}
    \label{fig:FirstArchi}
\end{figure}


\paragraph{Tâche :}

Nous avons testé cette architecture sur une tâche simple en simulation illustrée Fig.~\ref{fig:firstTask}. Dans cette tâche, l'Homme et le robot doivent enlever des objets d'une table et les mettre dans une boite rose. Au début de l’interaction, deux objets sont accessibles uniquement par le robot et un autre uniquement par l'Homme. La boite est accessible uniquement par le robot. Pour réaliser la tâche, l'Homme et le robot peuvent exécuter différentes actions (prendre un objet, ranger un objet, s'échanger un objet ou attendre).

Le comportement de l'Homme est simulé dans cette expérience. L'Homme est collaboratif : il exécute toutes les actions prévues pour lui dans HATP et participe à tous les échanges d'objets entrepris par le robot.

\begin{figure*}[!h]
\centering
	\subfigure[Situation initiale]{
        \centering
        \includegraphics[width=0.45\textwidth]{figs/Chapter7/initFirstTask.png}
       \label{subfig:initFirst}
   }\hfill
    %~
	\subfigure[Situation finale]{
        \centering
        \includegraphics[width=0.47\textwidth]{figs/Chapter7/endFirstTask.png}
       \label{subfig:endFirst}
   }
    \caption{Tâche utilisée pour tester la première architecture. L'Homme et le robot doivent enlever les objets de la table et les mettre dans la boite rose.}
    \label{fig:firstTask}
\end{figure*}


\paragraph{Résultats :}

Pour tester notre architecture, nous avons réalisé la tâche avec chaque expert seul dans un premier temps puis avec la combinaison des deux. La tâche a été réalisée en boucle dans un temps imparti. Le critère principal utilisé ici pour évaluer le système est le nombre de fois qu'il est capable de réaliser la tâche dans ce temps imparti. Les résultats pour 10 simulations de 30 minutes dans chaque condition peuvent être trouvés Fig.~\ref{subfig:rewardsFirstTask}. On peut observer une faible performance du MF seul (algorithme d'apprentissage) due à son manque de connaissances initial. La combinaison des deux experts a de bien meilleures performances bien qu'elles restent en dessous de celles d'HATP seul. En effet, la tâche étant simple à résoudre pour HATP, son plan est toujours optimal. Finalement, nous pouvons voir Fig.~\ref{subfig:weightsFirstTask} que la combinaison du MF et d'HATP permet au MF d'apprendre bien plus vite que quand il est seul.

\begin{figure*}[!h]
\centering
	\subfigure[Moyenne des récompense obtenues sur 10 simulations de 30 minutes (une récompense par tâche achevée).]{
        \centering
        \includegraphics[width=0.45\textwidth]{figs/Chapter7/rewardsFirstTask.pdf}
       \label{subfig:rewardsFirstTask}
   }\hfill
    %~
	\subfigure[Moyenne de l'évolution des poids de connexion du MF seul et avec HATP. Plus l'amplitude est élevée et plus le MF a appris qu'elle action exécuter.]{
        \centering
        \includegraphics[width=0.45\textwidth]{figs/Chapter7/WeightEvolFirstTask.pdf}
       \label{subfig:weightsFirstTask}
   }
    \caption{Performances de l'architecture testée comparées aux experts seuls.}
    \label{fig:resultsFirstTask}
\end{figure*}

Les premiers résultats obtenus montrent que la combinaison d'HATP et du MF permet d’accélérer l'apprentissage du MF. Cependant, la tâche étant très simple, HATP n'a pas de difficulté quand il décide seul. 

\subsubsection{Seconde architecture: les limitations}

Dans un second temps, nous avons amélioré l'architecture et l'avons testée sur une tâche plus complexe afin de démontrer l’intérêt du système.

\paragraph{Architecture :}

Comme l'un des principaux avantages du MF par rapport à HATP est son temps de calcul, nous avons modifié l'architecture comme représenté Fig.~\ref{fig:SecondArchi}. Dans cette version de l'architecture le meta-contrôleur est en amont des experts. Un expert ne sera activé uniquement que lorsque le meta-contrôleur choisira qu'il doit décider de la prochaine action. 

\begin{figure}[!h]
	\centering
    \includegraphics[width=0.8\textwidth]{figs/Chapter7/SecondArchi.png}
    \caption{Seconde architecture développée pour combiner les deux experts.}
    \label{fig:SecondArchi}
\end{figure}

Nous avons également introduit dans cette nouvelle architecture un nouveau critère pour la prise de décision du meta-contrôleur (précédemment aléatoire). Ce critère est basé sur le coût de chaque expert (temps à trouver une solution) et son erreur de prédiction. 

\paragraph{Tâche :}

Pour augmenter la complexité de la tâche, nous avons dans un premier temps augmenté sa combinatoire. Dans la nouvelle tâche il y a maintenant 6 objets qui doivent aller dans deux boites différentes en fonction de leur couleur. Ces objets sont initialement placés de manière aléatoire sur 7 emplacements possibles sur la table au début de la tâche comme Fig.~\ref{fig:SecondArchi}. De nouvelles actions sont également possibles pour le robot, il peut maintenant replacer un objet sur un des emplacements ou naviguer à une autre position près de la table.

\begin{figure}[!h]
	\centering
    \includegraphics[width=0.8\textwidth]{figs/Chapter7/InitSecondScenario1.png}
    \caption{Tâche utilisée pour tester la seconde architecture. L'Homme et le robot doivent enlever les objets de la table et les mettre dans les boites de même couleur.}
    \label{fig:SecondArchi}
\end{figure}

Nous avons également ajouté dans la tâche des difficultés géométriques qui ne peuvent pas être gérées par la planification (certains objets supposés accessibles ne peuvent en fait pas être attrapés par le robot). Finalement, nous avons également implémenté différents comportements pour l'homme simulé (plus ou moins coopératifs).

\paragraph{Résultats :}

Comme la tâche est plus complexe, nous avons légèrement augmenté le temps de simulation. Comme précédemment, nous avons testé le système avec chaque expert séparément puis avec la combinaison des deux. Logiquement, HATP présente de meilleurs résultats avec un Homme plus collaboratif et le MF présente de pauvres résultats seul. Cependant, nous n'avons pas réussi à avoir des résultats pour la combinaison des deux experts supérieurs à ceux d'HATP seul. Cela est dû au fait qu'avec une tâche plus complexe, l'effet d'accélération d'HATP sur l'apprentissage du MF n'est plus suffisant pour lui permettre d'apprendre une solution pour la tâche et donc lui permettre d'aider le système. 

Afin d'obtenir un système plus performant, de possibles améliorations seraient de retravailler l'algorithme d'apprentissage afin de mieux l'adapter au contexte et lui permettre d'apprendre plus rapidement. Une autre amélioration possible serait de chercher un nouveau critère d'arbitrage pour le meta-contrôleur. Enfin, il serait intéressant de permettre à HATP d'obtenir un retour de ce qu'apprend le MF afin d'adapter en ligne ses modèles de planification.


\begin{figure*}[!h]
\centering
	\subfigure[Moyenne des récompenses obtenues sur 10 simulations de 40 minutes pour HATP seul (une récompense par tâche achevée).]{
        \centering
        \includegraphics[width=0.45\textwidth]{figs/Chapter7/reward_HATP.pdf}
       \label{subfig:reward_HATP}
   }\hfill
    %~
	\subfigure[Moyenne des récompenses obtenues sur 10 simulations de 40 minutes pour le MF seul (une récompense par tâche achevée).]{
        \centering
        \includegraphics[width=0.45\textwidth]{figs/Chapter7/reward_MFalone.pdf}
       \label{subfig:reward_MFalone}
   }
    %~
	\subfigure[Moyenne des récompenses obtenues sur 10 simulations de 40 minutes pour la combinaison des deux experts (une récompense par tâche achevée).]{
        \centering
        \includegraphics[width=0.45\textwidth]{figs/Chapter7/reward_combo.pdf}
       \label{subfig:reward_combo}
   }
    \caption{Moyenne des récompenses obtenues dans chaque condition.}
    \label{fig:resultsFirstTask}
\end{figure*}

\newpage
\section{Conclusion}

Plusieurs contributions sont présentées dans ce manuscrit :
\begin{itemize}
\item Dans un premier temps, nous avons étudié les bases de l'action conjointe entre Hommes et comment ces principes s'appliquent à l'interaction Homme-robot afin de construire un superviseur pour la décision lors de l'action conjointe Homme-robot. 
\item Dans un second temps nous avons étudié comment améliorer la gestion des plans partagés par le robot. Nous avons d'abord étendu l'estimation des états mentaux de l'Homme par le robot au plan partagé, puis, nous avons amélioré les algorithmes de gestion du plan pour rendre le comportement du robot plus flexible. Enfin, nous avons évalué ces deux contributions en simulation et en conditions réelles grâce à une étude utilisateur.
\item Enfin, nous avons présenté deux autres contributions à l'interaction Homme-Robot. La première concerne la gestion de la tête du robot lors d'une tâche collaborative. La seconde cherche à combiner deux méthodes de prise de décision par le robot : la planification et l'apprentissage.
\end{itemize}

\selectlanguage{english}

\ifdefined\included
\else
\bibliographystyle{StyleThese}
\bibliography{These}
\end{document}
\fi


\bibliographystyle{StyleThese}
%\bibliographystyle{plain}
\bibliography{These}

\end{document}
